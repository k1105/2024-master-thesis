\chapter{関連研究}
産業革命、そしてコンピュータによる情報革命と、人々が道具や機械を発展させる中、人と機械の最適な関係づくりを目指すエルゴノミクス、インタラクティブシステムをより使いやすく設計するための人間中心設計といったアプローチが、一般的な思想として浸透した。

ヒューマンインターフェースにおいては、機械や情報処理が介在する高度で複雑な道具であっても、初歩的な道具にあったような作用と結果の対応関係のわかりやすさが問われ、理想のヒューマンインターフェースとは「使っている最中にはその道具自体を意識せずに身体の一部になったかのようになり、目的に集中できるようにすること」、すなわち「道具の透明化」の実現とされる。

インターフェース研究者の渡邊\cite{Watanabe2017}は、この理想を原理として実現する上で「自己帰属感(sense of self ownership)」の重要性を指摘した。自己帰属感とは、Gallagher\cite{Gallagher2000}が提唱するミニマルな自己感覚の一つであり、随伴性が高ければ道具などの身体以外の物に対しても発生すると考えられている。渡邊は、「道具の透明性」とはこの感覚がもたらすと考えている\cite{Watanabe2003}。自己帰属感に着目すれば、例えばマウスカーソルやスマートフォンは、操作時の指とグラフィックの追従性が高いことから自己帰属感が発生するため、「道具の透明化」が実現された理想的なインターフェースである、と説明できる。

このように自己帰属感は、人間中心設計の立場からヒューマンインターフェースを最適化するための重要な概念であるが、同時に認知科学の分野では、自己像の柔軟さを説明する概念としても用いられている。

こうした知見はロボティクスやHCIの分野にも応用される。稲見らによる自在化身体プロジェクトでは、

ここまでみてきたように、人間と機械の関係は、使い方に着目してシステムを最適化していくようなアプローチから、近年では拡張

\section{title}
\section{身体拡張・変換と身体性}
\section{Felsによる身体性のカテゴリー}
\section{本研究の位置付け}
しかしこれらでは、OOという観点を見落としているのではないか。それを扱う上で、本研究の\textit{grasp}というコンセプトを提示する。
graspとは、OOである。