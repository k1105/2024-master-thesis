\chapter{関連研究}
ラバーハンド錯覚実験などにみられるように、生来の肉体ではないものを自分の身体と錯覚してしまうような、人間の身体像の曖昧さに注目した研究が認知科学や心理学の領域で取り組まれてきた。人は自分の肉体のみを身体として知覚するのではないことが明らかになった中で、Gallagher\cite{Gallagher2000}は自己を構成する最小限の要素を「ミニマルセルフ」とし、それを構成する要素として「sense of ownership」と「sense of agency」を提唱した。こうした研究で蓄積された実証的知見は、インターフェースデザインや身体拡張の設計・評価へと応用されている。

インターフェース研究者の渡邊は、機械や情報処理が介在する高度で複雑な道具であっても、「使っている最中にはその道具自体を意識せずに身体の一部になったかのようになり、目的に集中できる」こと、すなわち「道具の透明化」という、ヒューマンインターフェースの理想を実現するための指針としてこの概念に注目する。渡邊は、例えばマウスカーソルやスマートフォンのような「操作時の指とグラフィックの追従性が高い」インターフェースには「自己帰属感(Gallagherのsense of ownershipに対する訳語)」が生じるとし、このことからこうしたインターフェースは「自身の一部、延長」として考えられ、「道具の透明化」が実現すると説明する\cite{Watanabe2013}。

また、自在肢プロジェクト、えくす手、親指の動きを肩の動きにマッピングさせる実験など、VRやロボティクスの分野でも取り組まれている研究では「sense of agency」や「sense of ownership」に基づく評価・設計が行われている。

しかし、これらの実践が注目するのは、「受動的」で「標準的」にもたらされる「ミニマルセルフ」ではないだろうか。

楽器が演奏できるときや自転車に乗れるようになるときのような、目的意識とその達成に向けた試行錯誤を通して獲得する「能動的」で「属人的」な「ミニマルセルフ」が注目されることは少ない。そうした経験を扱う上で、本研究の\textit{grasp}というコンセプトを提示する。

「grasp」は「把握」を意味する動詞・名詞だが、日本語でも物体を把持する手の動作であると同時に、概念について理解するときにも用いる。ここでの\textit{grasp}とは、「人がある対象に注意や目的意識を抱き、注意を向けた対象を確かめるため、または目的の達成に向けて試行を行う期間」のことを指す。手指の構造を変化させる本作は、このコンセプトに注目し、手指の変換をきっかけに起こる注意や、変換した手指とボールとの関係のもとで生じる目的意識が、体験者の中で自発的に生まれることを目指した。
本研究では、作品体験についてのインタビューを通してそれが実現していたことを示し、そのデザインプロセスと最終的な作品形態についての考察を通して、\textit{grasp}の設計に向けた指針となることを目指す。

\section{本研究の位置付け}
