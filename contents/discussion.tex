\chapter{考察}
\label{考察}
\ref{validation}章では、修士作品の体験についてそれぞれの参加者がどのように体験していたかを紹介した。本章では、\ref{nerai}節で示した本作の体験における仮説と照らし合わせた上で、作品における成功した点と失敗した点について考察する。

\section{Familiar / Strange}
参加者1、2、4では、変換をきっかけとして挙動を確認するための動き、すなわちResponseの動作が見られた。また、Responseの際に試した動きが、scene7以降で確認された体験者1、2のようすを踏まえると、試行期間である\textit{grasp}の期間が作品との関係性を変化させることに寄与していると考えられる。しかしその一方で、くの字のユニットが縦に積み上がったscene7の前後以外で、\textit{control}が生まれることは少なかった。そのため、この作品を通して\textit{grasp}こそが、Intimacyを高めることに寄与していたかどうかについて検証することは、難しいと考えた。

また、参加者2は最初、「グー」「チョキ」「パー」といった、手の形をもって意味をなす手の動作から、手の形ではなくなったことをきっかけに指だけを動かすような動作へと変化していたが、参加者3は、手の形でなくなってもフレームを取るような、手の形をもって意味をなすような動きをしていた。そして、途中から「反応がない」と感じられ、「飽きていた」とコメントしている。手指の形が切り替わっていくことは認識していても、それに合わせて自分の身体の動きを変化させることがなかったことは、作品から影響されることはなかったのではないかと考えられる。すなわち、認知的身体が更新されなかったことを意味する。本作は明確なタスクがあるわけではなく、自身での目的設定をしない限り注意を向ける対象が存在しない。controlが生まれることが少なかったことにも関連することだが、今回の作品では、目的意識を設定しづらいといった問題があったことが示唆された。

しかし少ないながらも目的意識を抱いていた参加者は、何かしらの「見立て」をしていたことがわかった。参加者4は、指が縦方向の動きのみに拘束されてパタパタと上下する形になったときに、ピアノの演奏のような身体感覚を想起していた。また参加者2は、「頭の中で既視感を作って」体験していたと振り返る。こうした「見立て」が生じることは、体験する人がそこに目的意識を見出して、対象を捉えようとすることの表れと考えられる。

「見立て」は形に起因する見立ても存在するが、参加者4が身体の動きから「ピアノ」を連想していたように、過去に体験したことのあるような身体動作がこの場所で再演されることを通して、デジャヴのような感覚を引き起こす、「運動の見立て」の可能性も存在する。このように「見立て」が可能であることが、今後体験の中で目的意識や注意を自分で作って体験することにつながるのではないかと考える。

さらに、「思い通りに動く」とは「連動性が担保されていること」ではないか、という仮説が新たに生じた。「思い通り」という言葉は、相手に命令する立場からすると、期待している結果と実際の挙動が一致している際に生じるが、「control」の状態下での「思い通り」とは、参加者2が体験を振り返るコメントにもあるように、「トラッキングができて」いて、「時差を極力なくし」た動きが「自分の思い通りの動きになってくれる状態」、すなわち「連動性」ではないかということが示唆された。その上で、「Belonging:帰属感」のようなIntimacyが生じるためには、その動きがさらに細かく制御できていることが重要ではないかと考えられる。

また、体験者4は、度々「何も考えてなかった」瞬間があったことを振り返った。特に、初めて指がバラバラになるscene1への遷移、そして手指の形にまた戻ってくるscene1からscene0への遷移の際である。こうした注意を向ける対象が存在しなくなる瞬間には、\textit{grasp}は消失していると解釈した。その原因として、先に言及した「見立て」の問題を挙げる。手指が手首のところから分割された限りでは、特別に注意を喚起するような体験にはなっていなかったのではないかと分析した。

\section{Relation}
参加者1、2、3いずれも、途中からマトの存在に偶然気づき、それ以降は「マトに当てる」ということを目的として体験していた。意図としては、「マトに当てる」ということ自体は体験者に必ずしも求めていた要素ではなかったが、マトというモチーフが「当てることを期待している」体験として意識させてしまったのではないかと考える。
また、参加者2の意見にあったように、「投げるときの反応が鈍」いことから「イライラ」、「もどかしい」といって振り返っていた。一方で、トラッキングの精度の問題が悪く作品体験が成立してないと判断した参加者4は、トラッキングが途切れ途切れであったにも関わらず、トラッキングが途切れてしまったこと自体に不快感を示さなかった。
こうしたことを踏まえると、体験における「もどかしさ」とは目的意識との関係にあり、トラッキングの精度やトラッキングの可否の問題ではないと考えられる。また、作品がどのように動くのかについて、実際の正しい振る舞いについて理解することよりも、何かしらの納得感が現れることの方が重要であることが示唆された。

さらに、参加者ごとに作品の体験における「気持ちよさ」が、目的意識との関係で変化していたことがわかる。
参加者4は、「気持ちよさを感じるポイントは、ゲームを進めていく中で変化してったけど、最初は「球を上げる」っていう目標になって、それを達成する気持ちよさを超えて、次の目標が見つかったときには、的に当てないと気持ちよくない」と振り返った。

一方で、参加者3は、手指の対応関係については認識していたと振り返ったが、体験の途中でそうした対応関係とは関係なく、手のひらを上に向けて掬い上げるような動きをしていた。慎重に画面の中で行われていることを観察した上で身体を動かすというよりは、自分の表現したいことを身振りでジェスチャーするような体験であったと言える。本作品は、手指の身体動作が全て認識されて動作するため、あらゆる動作を受け付けるという点で、探索をする上で適度に抽象的であると考える。ただしその一方で、慎重になることについては必ずしも必ずしも喚起出来ているわけではないということがわかった。しかし、その設計は慎重に行う必要がある。なぜなら注意を喚起する構造を設計することは、抽象度を高めることとトレードオフの関係となってしまうことがあるためである。

\section{総括}
2つの作品では、それぞれ異なる性質の\textit{grasp}を喚起することが出来ていたことが明らかとなった。しかしその一方で、\textit{belonging}や\textit{control}といった関係性が芽生えていたと判断できるような状況が少なく、\textit{grasp}こそが親密さを高める要素として寄与していたのかについては、検証することが難しいと考えた。

そこで、今後作品の体験を発展させる上で、\textit{control}や\textit{belonging}を発生させる可能性がある表現を、今回の分析を踏まえて列挙する。

\section{Felsの議論の限界}

\section{\textit{grasp}について}
\textit{grasp}について、本研究では次のように定義する。

\begin{quote}
  \textbf{人と対象との関係の中で、人が対象の中に注意や目的意識を抱きながら、意識的に試行する期間}
\end{quote}

graspは「把握」を意味する動詞でもあるが、ここで「動作」ではなく「期間」とした。その理由は、「意識的に試行する」とき、同時にその結果を受けて気づきを得たり、その気づきをもとに新たな関心を抱くといった、単に自分が行為しているだけではなく、対象から影響を受けながら次の行為が決まってくるようなフィードバックループの構造があると考えるためである。「grasp=把握」という言葉についても、単に「ものを掴む」という意味だけでなく、「理解」の意味があることは、対象について一方向に働きかけているのではない様子が現れているのではないだろうか。

\textit{grasp}とは例えば、ギターの習得過程において、弾きこなしたいフレーズを定め、それを達成するまでに試行錯誤をし、達成できるようになるまでの期間である。熟達した状態では、熟達する前とは違う視座でものごとを捉えられるようになり、また違う対象に注意が向くようになると、ギターと人との間に別の\textit{grasp}が芽生える。
またあるいは、\textit{grasp}の過程で、対象と向き合い続ける過程の中でその解像度が高まり、当初目指していたこととは違うことに興味を抱く(セレンディピティ)ことでも、別の\textit{grasp}が芽生える。

ここまで具体例を通してみてきたように、\textit{grasp}は、ギターと人との関係において一度だけ生じるのではなく、注目する対象が定まれば何度でも生じる。


\begin{quote}
  \begin{enumerate}
    \item 「見立て」が可能な作品は、さらに目的意識を設定しやすく\textit{control}の関係性が芽生えることが考えられる。そのためには積極的に「見立て」が生じる状況を設計するというアプローチもありうるが、プロトタイピングでもあったように作品体験の複雑度を上げるような展開のさせ方も考えられる。今回の「Familiar / Strange」では、トランジションの過程でこの「見立て」が特に起こり得ず、目的意識を抱きづらい状況が続いていたことが課題として挙げられる。
    \item 「マト」の存在は、全体的に強い目的意識を喚起させる表現となっていた一方で、体験者が目的意識を設定する余地をなくしてしまっていたことが考えられる。これは、「ゲーム性」を追及するのではない方法で目的意識をいかに設計していくのかという、遊びに関する議論とも接続する。前の要素とも関連するが、例えばオンスクリーン表現ではなく、フィジカルな表現で関係性の複雑度を上げるなど、注意を向ける対象の増やし方には検討の余地があると考える。
    \item 体験の中で、抱いていた関心に対して試行しきることがないままに体験が終了していたことがあった。トランジションは作品の構造を示し、段階的に注意を向けることに寄与していた一方で、徐々に展開されてしまうことが集中を途切れさせる原因にもなっていたのではないかということが課題として挙げられる。
  \end{enumerate}
\end{quote}