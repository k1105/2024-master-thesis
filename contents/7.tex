\chapter{まとめ(執筆中)}
本研究では、楽器のような、習得のための練習を経てようやく道具と人とのあいだにIntimacyが生じるという対象と人の関係性を捉えるため、「人がある対象に注意や目的意識を抱き、注意を向けた対象を確かめるため、または目的の達成に向けて試行を行う期間」として\textit{grasp}というコンセプトを立てた。

「身体性の変容」にそうしたコンセプトを体現するための契機があると考え、モーショントラッキングを用いた手指の異なる形状への変換表現について、プロトタイピングを行い、関節運動を踏襲した変換表現、1対1の対応関係が確認される変換表現、そして、変換の前後をモーフィングで表現することが、そうしたコンセプトを体現するための表現要素として機能するのではないか、という仮説を立て、修士作品「Graps(er)」として、制作した。
本作での体験について、インタビューを通して精緻に振り返り、コンセプトが達成できているかについて分析を行った。
結果、「」といった点でこのコンセプトが達成されていた一方で、「」の点ではうまくいかず、また個人差が顕著に見られるということがわかった。

こうした議論を展開させて、今後は設計の指針として応用できる可能性があるが、そのためにはすでにある他の議論を参照して、主にどういった分野に適用させていくか、その位置付けについて整理する必要がある。その契機として、\ref{考察}章では、シカールと青木淳の議論を挙げた。
