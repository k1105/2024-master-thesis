\chapter{まとめ}
\label{matome}
本研究では、手指の異なる形状への変換から「身体の中の他者性」を経験させる表現を通してこの問いについて探索し、修士作品《Grasp(er)》を制作した。さらに、その体験を説明するものとして\textit{grasp}という概念を定義した。

本研究の狙いは、\textit{grasp}の観点から人と道具、機械、あるいは人体の中にある他者性との関係における「人馬一体」のような関係を捉えることである。そして、そうした関係性はどのようにして引き出すことができるのかを提示することである。

作品の分析を通して、本作品の表現は\textit{grasp}を喚起するものとして成功した側面もあったが、手指の変換構造については注意深くとらえない限り、認識が変化しないという体験の余地も残されていたことが明らかとなった。
また、この作品の体験を通して\textit{grasp}は生じていたものの、Felsの\textit{control}や、\textit{belonging}が生じていたと考えられる行為は少なかった。これは、作品の複雑さに起因する上限の可能性も考えられるため、\textit{grasp}がこうした関係性へと導くきっかけとなることの反証であったとは言いづらいが、同時に\textit{grasp}こそがこうした関係を喚起するものであったと結論づけるには、表現における課題が残り、検証に至らなかったと考える。

そこで、前節では今後に向けた表現の方向性として、体験者の意見や分析を通して得られた「見立て」や「複雑度」、「モーフィング表現についての再検討」が重要であると考える。

\section{今後の展望}
本節では、今回の研究や作品に反映することができなかったが、今後この概念をより発展させていくにあたり、論じていく必要があると考える2つの論点について整理する。1つは、青木淳による「原っぱと遊園地」、もうひとつは、ミゲル・シカールの遊び論に関する議論である。

\subsection{「原っぱと遊園地」}
建築家の青木淳は、建築においてそこで行われることをあらかじめ決定されているような空間を「遊園地」とし、一方で行われることで空間が作られていくような空間を「原っぱ」と呼び、分類している。

\begin{quote}
  ともかく、廃校になった機能主義的小学校の空間は、ちょうど原っぱのように、人間にそれ に対するかかわり方の自由を与える。 原っぱとは、つまり空き地である。 宅地が造成され区画 される。これは人工的な営みである。 塀が築かれ、土地の形がきちんと確定される。一度は土地が均され雑草が刈り取られる。 そこまで行って、なにかの理由で、放置される。時間が経過して、 セイタカアワダチソウなどの雑草が、背丈ほど伸びてくる。そして、原っぱができ上がる。 更地というだけでは、原っぱではない。放置後の、適切な度合いの自然の遂行を必要とする。 
  廃校になった牛込原町小学校は、原っぱと同じく、人間の感覚とは一度は切れた決定ルールによって生成し、しかしその決定ルールが根拠を失った空間だったのである。そうして、偶然に、そこで人がつくることと、与えられる空間の規定力の対等が実現されていたのである。  
\end{quote}

こうした「原っぱ」的な空間においては、そこにいる人が目的設定をした上で注意を向ける余地が生まれやすい。これは建築という空間設計の話だけではなく、\textit{grasp}の生起を目指したプロダクトや体験を設計する上で重要な設計論であるととらえている。

\subsection{シカールの「ふざけた遊び」}
しかしこうした「原っぱ」的なプロダクトに対して、本研究が期待するような関わり方が生起するためには、メディアアーティストの久保田が「使いやすさ」を志向することではなく、「使いたさ」を志向することの重要性を説くように、「強い目的意識」が生じることが重要である。

\begin{quote}
  楽器は、リアルタイムに音を生成、コントロールするための道具である。そのインターフェイスの使いやすさが重要になるのはいうまでもない。 しかし、ピアノのインターフェイスは「はじめての人にも使えるように」あるいは「一目でわかるように」デザインされているだろうか?ギターは?トランペットやサックスの場合は?いずれの楽器も、 はじめての人にとっては音を出すだけでも四苦八苦の、使いにくい道具である。にもかかわらず、人はそんな使いにくい楽器に対して時間をかけてじっくりと向き合い、日々練習を重ね、少しずつスキルを向上させながら美しい調べを自在に奏でる夢を見る。そうした営みを支えているのは、人々の願いやビジョンである。(中略)「使いやすさ」のためにも、何よりもまず「使いたい」という願望が必要である。「どうやってやるか」ということよりも「何をやるか」のほうが先に来る。 豊富な機能を生かすも殺すもインターフェイス次第であることは間違いないが、その豊富な機能を使いたいと思えなければしょうがない。考えてみれば当然だ。 携帯電話のテンキーによる文字入力が良い例だ。何も工夫していないものが、結局は一番使いやすいインターフェイスになっている。重要なのは、表面的な改良ではなく、大きさや重さ、速度といった基本的な要目だ。妙な工夫をするよりもむしろ、シンプルであればあるほどスキルが活躍する余地が生まれる。ワンタッチで具体的なインターフェイスよりも、システマティックで抽象的なインターフェイスのほうが、多様な使用法とアウトプットを生み出すことができる。その名人芸と形容したくなるほどのスキルを生み出しているのは、人々の欲望だ。ここでも、欲望さえあれば、指が勝手に動いていく。技術革新の速度はそれなりに速いのかもしれないが、その気になった人間の適応力や柔軟性による変化の速度はもっと速い。人間の可能性は、底知れない。
\end{quote}

しかし、こうした目的意識の生起については、自発的に発生するものもあれば、コミュニティの単位などで創発する場合もあるのではないかと考えられる。

こうした創発について述べるのは、ミゲル・シカールによる遊び論で語られる、「ふざけた遊び」の性質を持った遊び心の発露であろう。

\begin{quote}
  遊びは存在のためのポータブルな道具である。ゲームに代表されるような遊びの「形式」よりも、「世界のうちに存在するモードの一種」としての遊び、つまり人間が人間としてあるあり方のひとつとしての遊びである。遊びは文脈に依存する。これは遊びが、物や環境やテクノロジーや人などからなる、その都度の「文脈」を使うかたちで生じるということだ。シカールの考えでは、もともと遊びのためにデザインされたものであるはずのゲームのルールですら、その本来の目的を無視して流用してしまうのが、遊び心という態度であり、遊びという「人間存在のモード」なのだ。  
\end{quote}

本研究が着目した\textit{grasp}のコンセプトは、こうした人間の感情の発露を期待する。その意味で、誰もが同じようにそれに価値を見出すような、一般的な利便性よりも、属人的な楽しさを志向したアプローチであると言えるだろう。今後は、これらの議論と関連付けながら、本研究において達成することが出来なかった、\textit{belonging}や\textit{control}の関係性を喚起することについて、さらなる検討を深めていくことを考えている。