\chapter{まとめ}
FelsのいうIntimacyの考えに基づいてこれらの事例を考えると、これらが扱うIntimacyの経験は、「錯覚」「コントローラブル」な経験として説明できる。ラバーハンド実験のような「錯覚」体験については、頭で理解していても抗えないほど強いIntimacyを引き出す体験となっている。一方で、楽器のような道具と人とのあいだには、習得のための練習を経てようやく、Intimacyが生じる。楽器習得に関して言えば、演奏したい曲や弾きこなしたいフレーズが想起できているのに、思うように扱うことができないことから、そのギャップに「もどかしさ」を経験することがある。このようなIntimacyは、必ずしも楽しさだけでなく、不快さを伴う経験でもある。その過程に「もどかしさ」を経験することが特徴的な、こうしたIntimacyを扱う研究は、まだ少ない。
こうした背景を踏まえて本研究では、\textit{grasp}というコンセプトを立てた。\\
そのコンセプトのもとで制作された修士作品「Grasp(er)」について、その体験をインタビューを通して精緻に振り返り、そうしたコンセプトが達成できているかについて分析を行った。
結果、「」といった点でこのコンセプトが達成されていた一方で、「」の点ではうまくいかず、また個人差が顕著に見られるということがわかった。
