\chapter*{謝辞}
清々しい気持ちでこのページを書いているつもりが、書いてみてようやく自分の至らない点や、自分自身まだわかりきっていない部分が露呈して、実際には軽く吐き気がしています。しかしそれらを整理して再構成するには時間があまりに足りないので、論文執筆を通して指導いただいたことをもとに、時々思い返しては自分自身で赤入れをしていこうと思います。\\
主査の桑久保 亮太教授、副査の平林 真実教授には、度重なるアドバイスと作品・論文の審査をしていただきました。概念を曖昧にして用いていた点についての指摘を受けて論点を整理していく中で、いっそう明確になりました。\\
主指導教員の小林 茂教授には、研究内容にかかわるアドバイスに加え、論文執筆に際してはパラグラフごとのまとまりとパラグラフ間のつながりについて、細かく指導していただきました。応えることのできなかった指摘も多いとは思いますが、教わった作法は今後しっかりと身体化(embodiment)させていきたいと思います。\\
小林ゼミの小南菜子さん、太向弘明さん、成瀬陽太さん、そして聴講の松井美緒さんには、素朴な疑問から鋭い指摘まで、同じ学生同士だからこその忖度のない、貴重な意見をたくさんいただきました。私は今年で卒業ですが、この経験をみなさんの研究活動に還元していくことができたらと思います。また合宿にもいきましょう\footnote{それと、論文執筆に追われていたため未遂に終わった忘年会をやりたいです。}。\\
最後に、インタビューにご協力いただいた皆様、展示の機会を設けてくださったFab Cafe Nagoya様、IAMAS22期生の皆様、そのほか多くの研究にご協力いただいた皆様\footnote{列挙したらキリがなく、ひとまとめにしてすみません。これを執筆している今、お世話になった人々が、走馬灯のように頭の中を駆け回っています。あなたのことを、忘れていません。お会いしたときに直接感謝を伝えさせてください。}に深く感謝申し上げます。

\addcontentsline{toc}{chapter}{謝辞}