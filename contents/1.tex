\chapter{はじめに}
\label{introduction}
% 1.1 はじめに
%--------------------------------------------------------1.1 はじめに

\section*{「使いづらさ」は悪か?}
産業革命、そしてコンピュータによる情報革命と、人々が道具や機械を発展させる中、人と機械の最適な関係づくりを目指すエルゴノミクス、インタラクティブシステムをより使いやすく設計するための人間中心設計といったアプローチが、一般的な思想として浸透した。「使いやすく」するための方法として、そのシステムが「どのように使われるのか」を定義し、その行為を先まわるように形態やシステムを決定する。例えば、D・Normanによるアフォーダンスは、ある行為を誘導するヒントとなるような形態を指し、「使いやすさ」を設計する重要な手段となっている。また、ISOによって提唱される人間中心設計の原則\cite{hcd}によれば、「ユーザ、タスク、環境の明確な理解に基づいたデザイン(1)」、「ユーザ中心の評価によるデザインの実施と洗練(3)」、そして「プロセスの繰り返し(4)」とあるように、システムの評価にユーザを交え、システム利用の中でのユーザの要求を理解し、それを明示するシステムへと設計解決策を漸次的に導き出すことが重要であるとされる。
このような方法論はヒューマンエラーを引き起こす悪しきデザインを是正してきた一方で、行き過ぎた「使いやすさ」の志向は、私たちにとって何か、根源的なものを奪っているのではないだろうか。

\section*{2つの「使いづらさ」}
私たちが道具やシステムの使用において「使いづらさ」を経験する理由には、「設計の問題」、「技量の問題」の2種類があると考える。

「設計の問題」というのは、Donald Normanが「誰のためのデザイン?」で問題に挙げていたように、人がそれを使用する上でエラーを頻発してしまうような、造形に由来する使いづらさのことだ。そして、こうした問題については人間中心設計、ユーザエクスペリエンスのデザインといった方法論によって解消されてきた。

一方で「技量の問題」とは、例えば楽器や自動車の運転のように、人間の身体のスケールやそこでの振る舞いに基づいてその造形は洗練されているにもかかわらず、それでもなお習得に時間がかかるような「使いづらさ」である。この種類の「使いづらさ」は時間をかけ、使い手である私たちが技量を身につけることで、「使いこなせる」ようになるものだ。その過程で私たちは、自らのやり方を模索し、試行錯誤する。ピアノを初めて弾くとき、奏者は手の大きさによる制約を受け、最初のうちは左右別々に指を動かすだけでも困難さを経験する。しかし自分に合ったやり方で演奏できるようになることに喜びがある。

この意味での「使いづらさ」は、たとえ不快であっても取り除かれるべきものではないだろう。しかし、これら2つの「使いづらさ」はその境界が曖昧で、特に2つ目の「使いづらさ」が生じる状況については、「技量が求められる」という説明だけでは掴みづらい。

そこで筆者は、「自分」と「自分ではないもの」の関係性において、それらが「一体」となるよう変化する過程に着目することで、2つ目の意味での「使いづらさ」を捉えた。
% そう主張する根拠として、1つはすでに挙げた通り、人間の喜びを奪ってしまうからで、もう1つは「人が変化すること」、そして「変化」を契機として、新しいことが生まれる機会を奪ってしまうからであると考える。

% ミュージシャンのスキャットマン・ジョン(ジョン・ポール・ラーキン)は、「吃音症」という発話障害を持ちながらも、自身の身体をコントロールできないその症状を逆手に取り、「テクノスキャット」という独自の歌唱法を開拓した。「吃音」という障害と向き合いながら、単に歌唱法を発明しただけでなく、個人の存在のしかたを創造したと言える。\\
% 産業革命、そしてコンピュータによる情報革命と、人々が道具や機械を発展させる中、人と機械の最適な関係づくりを目指すエルゴノミクス、インタラクティブシステムをより使いやすく設計するための人間中心設計といったアプローチが、一般的な思想として浸透した。「使いやすく」するための方法として、そのシステムが「どのように使われるのか」を定義し、その行為を先まわるように形態やシステムを決定する。例えば、D・Normanによるアフォーダンスは、ある行為を誘導するヒントとなるような形態を指し、「使いやすさ」を設計する重要な手段となっている。また、ISOによって提唱される人間中心設計の原則\cite{hcd}によれば、「ユーザ、タスク、環境の明確な理解に基づいたデザイン(1)」、「ユーザ中心の評価によるデザインの実施と洗練(3)」、そして「プロセスの繰り返し(4)」とあるように、システムの評価にユーザを交え、システム利用の中でのユーザの要求を理解し、それを明示するシステムへと設計解決策を漸次的に導き出すことが重要であるとされる。
% このような設計プロセスのもとで、「不安定さ」と向き合うような経験をすることが見落とされているのではないだろうか。
% アフォーダンスは過去の経験や私たちの身体構造によって形成される「直観」を拠り所にする。「こうしたい」というユーザの要求に先回りして応えるシステムが出来上がる。
% いずれも、疑問や不自由さを経験するよりも先にそれらを潰しておくような「いたれりつくせり」で「わかりやすい」システムだろう。\\

% \section*{わからなさと向き合う}

% ピアノを初めて弾くとき、奏者は手の大きさによる制約を受け、最初のうちは左右別々に指を動かすだけでも困難さを経験する。しかし、試行錯誤を経て自分に合ったやり方で演奏できるようになることに喜びがあり、パフォーマティブなのであろう。
% ミュージシャンのスキャットマン・ジョン(ジョン・ポール・ラーキン)は、「吃音症」という発話障害を持ちながらも、自身の身体をコントロールできないその症状を逆手に取り、「テクノスキャット」という独自の歌唱法を開拓した。「吃音」という障害と向き合いながら、単に歌唱法を発明しただけでなく、個人の存在のしかたを創造したと言える。\\
% 最初から使いやすく便利なものには、主体性を発揮する余地が少ない。試行錯誤から個人がやり方を見出すことで、高度なことが自在にできるようになる。またそれが、簡単に乗り越えられる経験ではないからこそ、乗り越えることに喜びがあり、パフォーマティブでもある。至れり尽くせりな道具を作るまでもなく、人々は所与の条件の中で自分なりのやり方を見出し、乗りこなしていくことができるのである。\\
% こうした背景から、私はもとの安定状態を手放し、別の安定状態を獲得する過程にある「一時的な不安定さ」に注目し、こうした経験を具現化するデザインについて、プロトタイピングを通して探索した。

\section*{「自分ではないもの」と一体化する}
「人馬一体」という言葉があるように、「自分」と「自分ではないもの」との間にそれらが一体となるような、なだらかで巧みな連携が行われることがある。これは人と自動車のあいだ、人とピアノのあいだにおける連携、一体感にも共通する。このような連携が達成されるためには、自分の欲求を相手にそのまま押し付けるだけでなく、多かれ少なかれ自分が柔軟になって、「自分ではないもの」と調和していこうと試みること、すなわち「自分」が「変化」する態度が求められる。これこそが、「自分」と「自分ではないもの」との関係性から捉えた「使いづらさ」である。

\section*{認知科学のEmbodimentとFelsのEmbodiment}
さて、このように「自分ではないもの」と一体化し、身体の延長のように経験される感覚については、認知科学やコンピュータ科学の理論用語として「Embodiment:身体化」と説明される。中でも参照されることの多いものとして、認知科学者・哲学者であるGallagher\cite{Gallagher2000}の「ミニマルセルフ」という考え方であり、ラバーハンド実験のような錯覚体験や、VR空間上のアバターに対する自己認識、さらにはマウスカーソルに対する「自身の延長のように感じる感覚」についての説明\cite{Watanabe2013}にもこの理論が用いられる。しかし、こうした文脈における「Embodiment」には、先に指摘した「自分」が「変化」する態度に注目した議論は少ない。

そこ本研究では主に、認知科学やコンピュータ科学の理論用語として用いられる「Embodiment」ではなく、オブジェクトと人との関係性に着目したSydney Felsによる「Embodiment」、さらに「Intimacy」を参照して、こうした「使いづらさ」に向き合う経験を説明することを試みた。前者のEmbodimentは主に、「自分ではないもの」が一体化している状態について、その因子を説明することが主であるが、FelsのEmbodimentは、人とオブジェクトとの関係について「一体化(Embodiment)していない状態」についても言及している。これは、最初に提起した「技量の問題」による「使いづらさ」について説明する上で有効な分類である。

\section*{「把握」による一体化}
Felsの分類に基づき、本研究では\textit{grasp}というコンセプトを定義した。「把握」を意味する動詞の「grasp」だが、日本語の「把握」と同様、「物理的にものを掴む」ことを意味するのと同時に「理解」の意味でも用いられる。身体的な試行錯誤の中から「自分ではないもの」との調和が生まれ、一体となっていく、そうした流れを捉えるため、この用語を用いた。\\
このコンセプトに基づき「手指の構造変換」を扱い、「自分」が「変化」する態度が現れることを目指した修士作品「Grasp(er)」を制作した。\\
本論文ではまず、関連研究や作品について、FelsのEmbodimentやIntimacyの視点から分類し、本研究の位置付けを示す。その上で、最終的な作品に至るまでに探索的に試みたプロトタイピングについて、試みたことの分類や取捨選択の過程を示すことで、コンセプトや表現の根拠とする。最後に、制作された修士作品が実際にはどのように体験されるのかについてインタビューを実施し、作品として何が達成されたのかについて考察すると同時に、今後の展望を述べる。

\section{本論文の構成}
第\ref{related_works}章ではFelsのEmbodimentを説明し、「身体化するまでの過程」から本研究の位置付けを整理し、先行研究や作品実践に対しても同じ視点から相違点の説明を試みる。

第\ref{graspについて}章では、本研究が提示するコンセプトである\textit{grasp}について詳細に説明する。その上で、Felsの分類との関係性や、このコンセプトにおける性質を述べ、最後に修士作品に対してどのようにこのコンセプトが適用されているかについて説明する。

第4章では、探索的な本作品の制作プロセスについて説明することで、本研究が提示するモデルのコンセプトがどのようにして得られたのかについてを説明し、また最終的な作品形態の詳細を述べる。

第5章では、そのようなねらいのもと制作された本作品が、実際どのように経験されるのかについての質的調査を行うため、Video Cued Recallという手法を用いて体験について振り返り、実際の体験がどのようなものであったのかを踏まえて、モデルとの関係性について考察する。

第\ref{考察}章では、行った調査の結果からどのような体験の作品であったかを振り返り、本作品が狙いとしていたこと、あるいはその範疇を超えて、作品として何が達成されたのかについて考察する。

第7章では、これまでの議論をまとめ、最後に今後の展望として、ここで制作を行ったモデルがそれ以外の議論とどのように関係し、今後どのような可能性を有するのかについて述べる。