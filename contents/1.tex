\chapter{研究概要}
\label{introduction}
本研究では、人間と道具や機械との関係において、人にとって道具や機械が「より使いやすく、有用に」あるという関係ではない、対象の他者性と向き合い、折り合いをつけていく経験を通して生まれる「一体感」を目指し、それがどのように生まれるのかについて探求する。この意味での「一体感」とは例えば、ピアノの演奏やバイクの運転など、人間が道具や機械の特性を理解し、使いこなすための能力を身につけることで生じる人間と対象との関係である。

この主題に、手指の変換を通して「身体の中の他者性」を経験させる表現を探索することから迫り、修士作品《Grasp(er)》を制作した。
また、人間と対象との関係について「Embodiment:一体化」の観点から分類したSydney Felsの4つのカテゴリやIntimacyの概念に基づき、本研究では\textit{grasp}という概念を提案した。この概念を用いて本作から「一体感」が生じるまでの過程についての仮説を立て、4名の作品体験を振り返ることから、それが実現しているかについて考察する。

\section{本研究の目的}
本研究の目的は、対象の他者性と向き合い、折り合いをつけていく経験を通して生まれる「一体感」について、それがどのように生じるのかについての詳細な説明と、そのために導入された概念である\textit{grasp}の妥当性を示し、そうした関係性が生じるものについての設計の指針を示すことである。

\section{本論文の構成}
本章では、研究の概要を示した。

第\ref{related_works}章では、研究背景について問題提起の形で示し、その上でリサーチクエスチョンを提示する。また、その問いを探索する切り口として本研究が着目した「手指の変換」について、その観点から取り組む動機、そして関連研究との相違点を説明することから研究の位置付けを明らかにする。

第\ref{about_grasper}章では、作品概要と、そこに至るまでのプロトタイピングの分析を通して、作品形態について説明する。

第\ref{graspについて}章では、本研究が提示するコンセプトである\textit{grasp}について詳細に説明する。その上で、Felsの分類との関係性と、この概念を用いることから修士作品の体験についてのねらいを説明する。

第\ref{validation}章では、そのようなねらいのもと制作された本作品が、実際どのように経験されるのかについての質的調査を行うため、Video Cued Recallという手法を用いて体験について振り返り、実際の体験がどのようなものであったのかを踏まえて、モデルとの関係性について考察する。

第\ref{考察}章では、行った調査の結果からどのような体験の作品であったかを振り返り、本作品が狙いとしていたこと、あるいはその範疇を超えて、作品として何が達成されたのかについて考察する。

第7章では、これまでの議論をまとめ、最後に今後の展望として、ここで制作を行ったモデルがそれ以外の議論とどのように関係し、今後どのような可能性を有するのかについて述べる。