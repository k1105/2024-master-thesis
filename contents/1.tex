\chapter{はじめに}
% 1.1 はじめに
%--------------------------------------------------------1.1 はじめに

\section*{使いやすさを設計する}
産業革命、そしてコンピュータによる情報革命と、人々が道具や機械を発展させる中、人と機械の最適な関係づくりを目指すエルゴノミクス、インタラクティブシステムをより使いやすく設計するための人間中心設計といったアプローチが、一般的な思想として浸透した。「使いやすく」するための方法として、そのシステムが「どのように使われるのか」を定義し、その行為を先まわるように形態やシステムを決定する。例えば、D・Normanによるアフォーダンスは、ある行為を誘導するヒントとなるような形態を指し、「使いやすさ」を設計する重要な手段となっている。また、ISOによって提唱される人間中心設計の原則\cite{hcd}によれば、「ユーザ、タスク、環境の明確な理解に基づいたデザイン(1)」、「ユーザ中心の評価によるデザインの実施と洗練(3)」、そして「プロセスの繰り返し(4)」とあるように、システムの評価にユーザを交え、システム利用の中でのユーザの要求を理解し、それを明示するシステムへと設計解決策を漸次的に導き出すことが重要であるとされる。


\section*{わからなさと向き合う}
このような設計プロセスのもとで排斥されるのは、「わからなさ」や「ままならなさ」ではないだろうか。
アフォーダンスは過去の経験や私たちの身体構造によって形成される「直観」を拠り所にする。「こうしたい」というユーザの要求に先回りして応えるシステムが出来上がる。
いずれも、疑問や不自由さを経験するよりも先にそれらを潰しておくような「いたれりつくせり」で「わかりやすい」システムだろう。

「わからなさ・ままならなさ」と対峙するとき、私たちは自らのやり方を模索し、試行錯誤する。
ピアノを初めて弾くとき、奏者は手の大きさによる制約を受け、最初のうちは左右別々に指を動かすだけでも困難さを経験する。しかし、試行錯誤を経て自分に合ったやり方で演奏できるようになることに喜びがあり、パフォーマティブなのであろう。
ミュージシャンのスキャットマン・ジョン(ジョン・ポール・ラーキン)は、「吃音症」という発話障害を持ちながらも、自身の身体をコントロールできないその症状を逆手に取り、「テクノスキャット」という独自の歌唱法を開拓した。「吃音」という障害と向き合いながら、単に歌唱法を発明しただけでなく、個人の存在のしかたを創造したと言える。

最初から使いやすく便利なものには、主体性を発揮する余地が少ない。試行錯誤から個人がやり方を見出すことで、高度なことが自在にできるようになる。またそれが、簡単に乗り越えられる経験ではないからこそ、乗り越えることに喜びがあり、パフォーマティブでもある。至れり尽くせりな道具を作るまでもなく、人々は所与の条件の中で自分なりのやり方を見出し、乗りこなしていくことができるのである。

こうした背景から、私はもとの安定状態を手放し、別の安定状態を獲得する過程にある「一時的な不安定さ」に注目し、こうした経験を具現化するデザインについて、プロトタイピングを通して探索した。

\section*{「操作」に対する「把握」}
そうした探索の中で、私は「操作」の前にある「把握」という身体動作に注目した。

\section{本研究の目的}
\section{本論文の構成}
第2章では、関連研究と
第3章では、
第4章では、
第5章では、
