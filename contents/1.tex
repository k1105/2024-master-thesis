\chapter{はじめに}
\label{introduction}
% 1.1 はじめに
%--------------------------------------------------------1.1 はじめに

\section*{使いやすさを設計する}
産業革命、そしてコンピュータによる情報革命と、人々が道具や機械を発展させる中、人と機械の最適な関係づくりを目指すエルゴノミクス、インタラクティブシステムをより使いやすく設計するための人間中心設計といったアプローチが、一般的な思想として浸透した。「使いやすく」するための方法として、そのシステムが「どのように使われるのか」を定義し、その行為を先まわるように形態やシステムを決定する。例えば、D・Normanによるアフォーダンスは、ある行為を誘導するヒントとなるような形態を指し、「使いやすさ」を設計する重要な手段となっている。また、ISOによって提唱される人間中心設計の原則\cite{hcd}によれば、「ユーザ、タスク、環境の明確な理解に基づいたデザイン(1)」、「ユーザ中心の評価によるデザインの実施と洗練(3)」、そして「プロセスの繰り返し(4)」とあるように、システムの評価にユーザを交え、システム利用の中でのユーザの要求を理解し、それを明示するシステムへと設計解決策を漸次的に導き出すことが重要であるとされる。


\section*{わからなさと向き合う}
このような設計プロセスのもとで、「わからなさ」や「ままならなさ」と向き合うような経験をすることが見落とされているのではないだろうか。
アフォーダンスは過去の経験や私たちの身体構造によって形成される「直観」を拠り所にする。「こうしたい」というユーザの要求に先回りして応えるシステムが出来上がる。
いずれも、疑問や不自由さを経験するよりも先にそれらを潰しておくような「いたれりつくせり」で「わかりやすい」システムだろう。\\
「わからなさ・ままならなさ」と対峙するとき、私たちは自らのやり方を模索し、試行錯誤する。
ピアノを初めて弾くとき、奏者は手の大きさによる制約を受け、最初のうちは左右別々に指を動かすだけでも困難さを経験する。しかし、試行錯誤を経て自分に合ったやり方で演奏できるようになることに喜びがあり、パフォーマティブなのであろう。
ミュージシャンのスキャットマン・ジョン(ジョン・ポール・ラーキン)は、「吃音症」という発話障害を持ちながらも、自身の身体をコントロールできないその症状を逆手に取り、「テクノスキャット」という独自の歌唱法を開拓した。「吃音」という障害と向き合いながら、単に歌唱法を発明しただけでなく、個人の存在のしかたを創造したと言える。\\
最初から使いやすく便利なものには、主体性を発揮する余地が少ない。試行錯誤から個人がやり方を見出すことで、高度なことが自在にできるようになる。またそれが、簡単に乗り越えられる経験ではないからこそ、乗り越えることに喜びがあり、パフォーマティブでもある。至れり尽くせりな道具を作るまでもなく、人々は所与の条件の中で自分なりのやり方を見出し、乗りこなしていくことができるのである。\\
こうした背景から、私はもとの安定状態を手放し、別の安定状態を獲得する過程にある「一時的な不安定さ」に注目し、こうした経験を具現化するデザインについて、プロトタイピングを通して探索した。

\section*{わからなさの契機としての身体変容}
そうした探索の中で、私は「変容」に注目した。慣れ親しんでいる身体性が変化される経験は、根本的な部分を不安定にすると同時に、車椅子に乗って生活する人にとって、義足を提供することが必ずしも正解ではないように、変容に対しても人は柔軟に乗り越えて、新しい身体性を獲得できるのではないかと考えたからである。しかし、本研究が着目するような、「わからなさ」の経験について捉える議論は、新しい技術と人との接点であるインターフェースについて考えるHCIの分野では、あまり多く語られることはない。そこで、Sydney Felsによる「Intimacy」と「Embodiment」の用語を参照し、こうしたわからなさと向き合う経験を説明する\textit{grasp}というコンセプトを定義した。

\section{本研究の目的}
そこで本研究では、「わからなさ」と向き合う体験を捉え、体現する作品制作に向けたプロトタイピングとデスクリサーチを通して\textit{grasp}というコンセプトを立て、それを適用して制作した作品「Graps(er)」の制作、そしてVideo Cued Recallによる作品体験の分析から、このコンセプトが体現されていることについて示す。最後に、今後の展望として、このコンセプトが他のコンセプトとどのように関連するか、そしてどのような可能性があるのかについて考察し、このコンセプトに着目することの意義について論じる。

\section{本論文の構成}
第2章では、柔軟な身体性に着目した新しいヒューマンインターフェースデザインにおける実践について整理し、本研究の位置付けを示す。同時に、Felsによる身体性のカテゴリーの分類に基づき、\textit{grasp}というモデルによって本研究の取り組みを説明する。

第3章では、本研究が提示するコンセプトである\textit{grasp}について詳細に説明する。その上で、Felsの分類との関係性や、このコンセプトにおける性質を述べ、最後に修士作品に対してどのようにこのコンセプトが適用されているかについて説明する。

第4章では、探索的な本作品の制作プロセスについて説明することで、本研究が提示するモデルのコンセプトがどのようにして得られたのかについてを説明し、また最終的な作品形態の詳細を述べる。

第5章では、そのようなねらいのもと制作された本作品が、実際どのように経験されるのかについての質的調査を行うため、Video Cued Recallという手法を用いて体験について振り返り、実際の体験がどのようなものであったのかを踏まえて、モデルとの関係性について考察する。

第6章では、行った調査の結果からどのような体験の作品であったかを振り返り、本作品が狙いとしていたこと、あるいはその範疇を超えて、作品として何が達成されたのかについて考察する。

第7章では、これまでの議論をまとめ、最後に今後の展望として、ここで制作を行ったモデルがそれ以外の議論とどのように関係し、今後どのような可能性を有するのかについて述べる。