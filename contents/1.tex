\chapter{序論}
% 1.1 はじめに
%--------------------------------------------------------1.1 はじめに
\section{はじめに}
こんにち、一般に「直感的」で「使いやすい」ことは良いことであるされる。人々を取り巻く道具やインターフェースは、「人間中心設計」に基づき、「直感的」で「使いやすい」ものが目指される。人間中心設計とは、「システムの使い方」を人間工学やユーザビリティの知識と技術を用いて、より使いやすいシステムに最適化するアプローチである。「ユーザエクスペリエンスを向上させる」というときも、「使いやすさ」を目指すことが「向上」とされる。また、インターフェース研究者の渡邊は、自著「融けるデザイン」において、技術が個人の能力を拡張する上で、直感的に制御できるインターフェースの重要性を説いている。

確かに「使いやすさ」「直感的であること」が重要な場合もあるが、そうした経験のみが一辺倒に良しとされるべきではない。「使いやすさを設計する」ことは、予め定められた用途に行為を最適化することであり、これは人の行為や目的意識を限定するものだからである。

与えられた条件の中で自らのやり方を模索し「試行錯誤」する経験も重要であると考える。例えばピアノでは、奏者は手の大きさによる制約を受け、最初のうちは左右別々に指を動かすだけでも困難さを経験するが、ピアノの形を変えることはしない。試行錯誤を経て自分に合ったやり方で演奏できるようになったときの喜び、そして演奏中に独自のニュアンスを発見し、それが洗練されることで新たな奏法が生まれると考えられる。また、ミュージシャンのスキャットマン・ジョン(ジョン・ポール・ラーキン)は、「吃音症」という発話障害を持ちながらも、自身の身体をコントロールできないその症状を逆手に取り、「テクノスキャット」という独自の歌唱法を開拓した。「吃音」という障害と向き合いながら、単に歌唱法を発明しただけでなく、個人の存在のしかたを創造したと言える。

これらの例から、「試行錯誤」は人間にとって最小化されるべき無機質な経験ではないと考えられる。最初から使いやすく便利なものには、主体性を発揮する余地が少ない。試行錯誤から個人がやり方を見出すことで、高度なことが自在にできるようになる。またそれが、簡単に乗り越えられる経験ではないからこそ、乗り越えることに喜びもある。そして、道具や体験の設計に対しては、設計する人が想定する目的や範囲を超えた創発が個人から起こる可能性がある。したがって、ある目的に適合するよう最適化するだけでなく、試行錯誤を経験する余地を残すアプローチも重要であると言えるだろう。

\section{本研究の目的}
\section{本論文の構成}