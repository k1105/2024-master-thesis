\chapter{研究概要}
\label{introduction}
本研究では、人間と道具や機械との関係性に焦点を当て、単なる使いやすさや分かりやすさを超えた、深い一体感について探求する。これは例えば、ピアノの演奏やバイクの運転など、人間が道具や機械に適応し、それらと一体化することである。人間と対象との一体化を分類したSydney Felsの「Embodiment」の概念に基づき、\textit{grasp}という概念を提案し、これを用いて一体化をどのように設計できるかについて、修士作品《Grasp(er)》の制作プロセスを通じて探求する。

\section{本研究の目的(執筆中)}
人間と道具や機械との一体化に関する新たな理解を深めることを目的とする。

\section{本論文の構成}
本章では、研究の概要を示した。

第\ref{related_works}章ではFelsのEmbodimentを説明し、「身体化するまでの過程」から本研究の位置付けを整理し、先行研究や作品実践に対しても同じ視点から相違点の説明を試みる。

第\ref{graspについて}章では、本研究が提示するコンセプトである\textit{grasp}について詳細に説明する。その上で、Felsの分類との関係性や、このコンセプトにおける性質を述べ、最後に修士作品に対してどのようにこのコンセプトが適用されているかについて説明する。

第4章では、探索的な本作品の制作プロセスについて説明することで、本研究が提示するモデルのコンセプトがどのようにして得られたのかについてを説明し、また最終的な作品形態の詳細を述べる。

第5章では、そのようなねらいのもと制作された本作品が、実際どのように経験されるのかについての質的調査を行うため、Video Cued Recallという手法を用いて体験について振り返り、実際の体験がどのようなものであったのかを踏まえて、モデルとの関係性について考察する。

第\ref{考察}章では、行った調査の結果からどのような体験の作品であったかを振り返り、本作品が狙いとしていたこと、あるいはその範疇を超えて、作品として何が達成されたのかについて考察する。

第7章では、これまでの議論をまとめ、最後に今後の展望として、ここで制作を行ったモデルがそれ以外の議論とどのように関係し、今後どのような可能性を有するのかについて述べる。