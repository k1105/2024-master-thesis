\chapter{修士作品 《Grasp(er)》}
\label{about_grasper}
本章では、修士作品《Grasp(er)》の概要と、その制作プロセスを説明する。

\section{作品概要}
\begin{figure}[H]
  \centering
  \includegraphics[width=15cm]{img/thumbnail.png}
  \caption{修士作品《Grasp(er)》}
  \label{grasper}
\end{figure}

《Grasp(er)》は、「Familiar / Strange」と「Relation」という二つから構成された作品群である。手の形が大きく変化したり、その変化した手指を使った細かな操作が求められる中で芽生える、身体の他者性に対する注目を扱った作品である。

最初はうまくいかず「もどかしさ」を感じながらも、試行錯誤や意識的な身体動作を行う期間を経て折り合いをつけていく過程に、「変化」が起こる。高度な道具の能力を享受するだけでなく、\textit{grasp}を通して、人が能力を身につけるという側面に着目して、作品名を《Grasp(er)》とした。

以下では、本作を構成する2つの作品について説明する。

% これらに共通するのは、\textit{grasp}の中で個人が目的意識や興味を抱くことで、\textit{grasp}が連鎖的に生じることである。

% 制作者は「このようなことをしてほしい」という行為の中身を設計せず、体験する個人がその中で次々と注目する対象を見出すことによって、個人が行為を創造していくことを目指している。このため、「手指の構造や、手指を取り巻く環境を変化させることで、手指の運動に注目する構造」を作ることに取り組んだ。このときの「注目」が起点となって\textit{grasp}の期間が生じるが、その先々で起こる体験は個人に委ねられ、明示的な目的は設定されていない。

\subsection*{Familiar / Strange}
\begin{figure}[H]
  \centering
  \includegraphics[width=15cm]{img/fs-02.png}
  \caption{Familiar / Strange}
  \label{fig:familiar_strange}
\end{figure}
「Familiar / Strange」は、手指の配置や関節の数・位置が次々と変化していく作品で、手の運動に対する再注目が起きることを目指している。作品は最初、手の形がそのまま現れた状態から開始し、指の並びや関節の配置の変化が起きる。一本一本の指がくの字の形をして積み上げられる様子をピークに、逆順に変化が巻き戻され、再び元の手の形に戻るという、3分10秒で1ループの構造となっている。


変換の過程は、イージングやゴム紐が切れた時のような振動を伴う動きによって補完される。シーン遷移を説明する図を以下に示す。
\begin{figure}[H]
  \centering
  \includegraphics[width=15cm]{img/placeholder.png}
  \caption{Familiar / Strangeにおけるシーン遷移}
  \label{fig:diagram_familiar_strange}
\end{figure}

\subsection*{Relation}
\begin{figure}[H]
  \centering
  \includegraphics[width=15cm]{img/relation.png}
  \caption{Relation}
  \label{fig:relation}
\end{figure}
「Relation」は、変化した手指を取り巻く関係が次々と変化していく作品で、手と直接的に制御できないボールの関係性におけるGraspに焦点を当てている。1つ目の作品と同様、最初は手が表示された状態から始まり、段階的に並び替えられた手指を覆う皮膜が現れ、ボールが現れる。最後はマトが現れ、的を取るたびにボールの大きさが小さくなり、3つ連続して取ると、皮膜は消え、再び元の手の形に戻る。

トラッキングされた手指の位置が、手首から指ごとに分割され、左端から右へ、左手の小指から親指、そして右手の親指から小指の順に整列される。しばらくすると、指先以外の運動が捨象され、残された指先を結ぶ線が現れる。ここで、現れた線によって再び全ての指が1つのまとまりとして統合されることになるが、その線は後に現れるボールに対して衝突判定が適用される、新しい構造の手指を覆う皮膜のような機能を有する。皮膜のある領域を外れるとボールは落下するが、そのあとは再び画面の中央にボールが出現する。

さらに一定時間が経過すると、皮膜の上方に白い点:マトが現れる。マトに対してボールを当てると、ボールは一回り小さくなり、ボールを落とさずに合計3回マトに当てることでボールは消失し、皮膜が現れるときと逆の順序を辿って画面の中の手は再びもとの形状に戻る。

この作品は、1つめの「Familiar / Strange」と異なり、構造が変化した手指と、ボールという直接的に動かすことができない対象との関係性に注意が向けられることをねらいとした。

\section{ライブラリの開発}
本作品を構成するにあたり、基本的な関数をまとめたライブラリを開発した。ライブラリには、円滑に体験するための補完処理を実装している。
具体的には、ガウシアンフィルターによる平滑化処理、トラッキングが途切れた際の例外処理の2つである。

\subsection{ガウシアンフィルターによる平滑化処理}
推定精度の問題から、モデルより推定される姿勢情報をそのまま出力すると、手指を動かしていなくても小刻みに振動したり、一時的なフレームレートの低下に起因してスムーズに動作していないように感じることがある。\\
そこで、体験者にフィードバックする際に使用する姿勢情報は、前後2フレーム分のフレーム情報にガウシアンフィルターを適用した平滑化処理を実装している。ただし、トラッキングが開始した直後は5フレーム分のフレーム情報を使用することができないため、この場合は取得できる限りのフレーム情報を用いて同じ処理を行なっている。そのため以下では、各フレーム情報に対する重みづけと、それを用いて体験者に提示される姿勢情報を求める上での一般式を示す。
モデルより推定された最新の姿勢情報を\(P_{n}\)、出力されている姿勢情報を\(S\)とすると、
  % 平滑化フレーム S の定義
  \begin{equation}
    S = \sum_{i=-2}^{2} w_i' \cdot P_{n+i}
    \end{equation}

ここで、\(w_i'\)は正規化されたガウシアンフィルタの重みを表す。正規化前の重み\(w_i\)は、
\begin{equation}
  w_i = \frac{1}{\sqrt{2\pi}\sigma} e^{-\frac{i^2}{2\sigma^2}}
  \end{equation}

正規化された重み\(w_i'\)は、
  % 重みの正規化
  \begin{equation}
  w_i' = \frac{w_i}{\sum_{j=-2}^{2} w_j}
  \end{equation}
と表現される。
この処理のため、最良時で60fps程度で取得される姿勢情報は、慢性的に0.3sほどの遅延を伴って体験者にフィードバックされることになる。

\subsection{トラッキングが途切れた際の例外処理}
体験時、環境光や、手指を動かす範囲や速度の関係から、トラッキングが途切れることがある。素早い動きをしている最中に1フレームでも途切れると円滑に体験することができないため、この時は例外的に、トラッキングが途切れる直前のフレーム情報で失われたフレーム情報を埋め合わせる処理を実装した。また、トラッキングが途切れていることに起因する不快感は、本作品の体験外の問題なので、手指の動きがトラッキングできていない状態を視覚的にフィードバックするため、トラッキング不能時に塗りつぶしを透過する視覚効果を実装した(\ref{fig:track_true}, \ref{fig:track_false})。

\begin{figure}[htbp]
  \begin{minipage}[b]{0.5\linewidth}
    \centering
    \includegraphics[keepaspectratio, width=7cm]{img/track_true.png}
    \caption{トラッキングが正常にできているとき}
    \label{fig:track_true}
  \end{minipage}
  \begin{minipage}[b]{0.5\linewidth}
    \centering
    \includegraphics[keepaspectratio, width=7cm]{img/track_false.png}
    \caption{トラッキングに失敗しているとき}
    \label{fig:track_false}
  \end{minipage}
\end{figure}