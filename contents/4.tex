\chapter{修士作品 Grasp(er)}
\section{概要}
ここまで、身体性の変容をその契機として「わからなさ」「ままならなさ」を経験することについてFelsの分類に基づいて整理を行い、前の\ref{define_model}章で\textit{grasp}の定義を行った。

本章では、そうしたコンセプトに辿り着くまでに行ったプロトタイピングについて整理すると共に、作品の概要を述べる。また、本章の最後には、本作においてこの\textit{grasp}のモデルがどのように作品形態に実装されているかについて説明する。

\section{制作プロセス}
本節では、最終的な作品形態に至るまでのプロトタイピングについてと、その過程で構築されたライブラリの開発について説明する。
\subsection{プロトタイピング}
本作品の形態に至るまでに、総計60パターンのプロトタイプ制作を行なった。プロトタイピングにおける目的を時系列で3段階に大別すると、「コンセプトメイキング」、「変換表現によって錯覚する「手触り」についての発散」、そして「変換表現で異なる身体動作が引き出される要素の特定」である。
\subsubsection{コンセプトメイキング}
コンセプトメイキングについては、複数回のプロトタイプの発表を通して展示
\subsubsection{変換表現によって錯覚する「手触り」についての発散}
\subsubsection{変換表現で異なる身体像が引き出される表現への収束}
その一方で、円の半径に関節の動きをマッピングさせるような表現については、縮んだり、膨らんだりする動きには共感しづらい。これらは「エフェクト的」に経験されるため採用しなかった。\\
さらに、1つの動きを複製して円形に配置したり、フラクタル的に配置するパターンを排除した。これについては、グラフィックデザイナーの女性が体験した際「構造としては緻密であるのに、動きは単純であるように感じる」と意見した。その理由として彼女は、「対称性が前面に出ているため、シンボルとしての印象を強く感じてしまう」「意識していないところも同時に動いている感覚があるため、自分の動きだと思えない」からではないかと推測した。
また、関節を折り曲げる動き
関節を折り曲げる動き
\subsection{ライブラリの開発}
本作品を構成するにあたり、基本的な関数をまとめたライブラリを開発した。ライブラリには、

\section{作品構成}
一方で、プロトタイピングを通して同定した変換表現を通して生じた異なる身体像に対する共感覚のようなものは、変換された手を用いた作業に移行した瞬間に、注意が向きづらくなるといった課題が起こった。そこで、最終的な作品としてはその2つを分けて構成した。

\subsection*{Familiar / Strange}
「Familiar / Strange」は、手指の配置や関節の数・位置が次々と変化していく作品で、手の運動に対する再注目が起きることを目指している。作品は最初、手の形がそのまま現れた状態から開始し、指の並びや関節の配置の変化が起きる。一本一本の指がくの字の形をして積み上げられる様子をピークに、逆順に変化が巻き戻され、再び元の手の形に戻るという、3分10秒で1ループの構造となっている。

\subsection*{Relation}
「Relation」は、変化した手指を取り巻く関係が次々と変化していく作品で、手と直接的に制御できないボールの関係性におけるGraspに焦点を当てている。1つ目の作品と同様、最初は手が表示された状態から始まり、段階的に並び替えられた手指を覆う皮膜が現れ、ボールが現れる。最後は的が現れ、的を取るたびにボールの大きさが小さくなり、3つ連続して取ると、皮膜は消え、再び元の手の形に戻る。
トラッキングされた手指の位置が、手首から指ごとに分割され、左端から右へ、左手の小指から親指、そして右手の親指から小指の順に整列される。しばらくすると、指先以外の運動が捨象され、残された指先を結ぶ線が現れる。ここで、現れた線によって再び全ての指が1つのまとまりとして統合されることになるが、その線は後に現れるボールに対して衝突判定が適用される、新しい構造の手指を覆う皮膜のような機能を有する。皮膜のある領域を外れるとボールは下方に落下するが、そのあとは再び画面の中央にボールが出現する。
さらに一定時間が経過すると、皮膜の上方に白い点が現れる。白い点に対してボールを当てると、ボールは一回り小さくなり、ボールを落とさずに合計3回白い球に当てると、ボールは消失し、皮膜が現れるときと逆の順序を辿って画面の中の手は再びもとの形状に戻る。


この作品は、1つめの「Familiar / Strange」と異なり、構造が変化した手指と、ボールという直接的に動かすことができない対象との関係性に注意が向けられることをねらいとした。
\section{本作品におけるモデルの適用}