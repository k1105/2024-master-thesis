\chapter{\textit{Grasp}について}
\label{graspについて}

\section{概要}
第\ref{about_grasper}章では、作品概要と、そこに至るまでのプロトタイピングの分析を通して、作品形態について説明した。

第\ref{graspについて}章では、本研究が提示するコンセプトである\textit{grasp}について詳細に説明する。その上で、Felsの分類との関係性と、この概念を用いることから修士作品の体験についてのねらいを説明する。

\section{\textit{grasp}の定義}
\label{grasp_difinition}
\textit{grasp}について、本研究では次のように定義する。

\begin{quote}
  \textbf{人と対象との関係の中で、人が対象の中に注意や目的意識を抱きながら、意識的に試行する期間}
\end{quote}

graspは「把握」を意味する動詞でもあるが、ここで「動作」ではなく「期間」とした。その理由は、「意識的に試行する」とき、同時にその結果を受けて気づきを得たり、その気づきをもとに新たな関心を抱くといった、単に自分が行為しているだけではなく、対象から影響を受けながら次の行為が決まってくるようなフィードバックループの構造があると考えるためである。「grasp=把握」という言葉についても、単に「ものを掴む」という意味だけでなく、「理解」の意味があることは、対象について一方向に働きかけているのではない様子が現れているのではないだろうか。

\textit{grasp}とは例えば、ギターの習得過程において、弾きこなしたいフレーズを定め、それを達成するまでに試行錯誤をし、達成できるようになるまでの期間である。熟達した状態では、熟達する前とは違う視座でものごとを捉えられるようになり、また違う対象に注意が向くようになると、ギターと人との間に別の\textit{grasp}が芽生える。
またあるいは、\textit{grasp}の過程で、対象と向き合い続ける過程の中でその解像度が高まり、当初目指していたこととは違うことに興味を抱く(セレンディピティ)ことでも、別の\textit{grasp}が芽生える。

ここまで具体例を通してみてきたように、\textit{grasp}は、ギターと人との関係において一度だけ生じるのではなく、注目する対象が定まれば何度でも生じる。
% ここで、\textit{grasp}は人と対象の関係の中で、人が注意を向ける「対象」ごとに別の\textit{grasp}があると言えそうだが、注意を向けている「対象」がなんであるか、断言できなかったり、本人も判然としないこともある。そのため、

\section{Felsの議論との関係性}
ここでは\textit{grasp}が、FelsのEmbodimentとどう関係するのかについて説明する。
先の節\ref{grasp_difinition}に挙げたように、人は\textit{grasp}を通して、あるいはその過程で、対象から影響を受けながら次の試行が形作られていく。この時の「試行」には、挙動を確かめるような動作、すなわちFelsの「Response」もあれば、試行を通して得られた結果をもとに、何かに考えを巡らす「Contemplation」も含まれる。こうした試行の積み重なりから、人と対象のあいだに「Control」や「Belonging」の関係、すなわちFelsの意味でのEmbodimentが生じるのではないか。

つまりこの概念は、折り合いをつけていくまでの「間」に行われていることについて語ることを可能にするものである。\textit{grasp}の様相を捉えることで、現状からEmbodimentが達成されるまでのあいだに欠けているものが何であるかを語る術を得られるのではないか、と考える。

\section{\textit{grasp}を踏まえたIntimacyが起こるまでの過程についての仮説}
\textit{grasp}が、本研究の関心である「対象からの影響も受けつつ(Belonging)、相互の折り合いをつけながら生まれる一体感(Intimacy)が起こる」までの過程に必要ではないか、と考える。

そしてそのためには、注意を向ける対象や目的意識を抱く対象が変わりながらも\textit{grasp}が継続していく体験が良いのか、それとも注意を向ける対象は変わらず、1つのものに対する目的意識が長く継続し、\textit{grasp}も長い体験が良いのか。このいずれであるかが判別することができれば、より詳細にこの体験を説明することができると考えた。

\section{\textit{grasp}を用いた《Grasp(er)》の説明}
本作品を\textit{grasp}の観点から、どのような関係が芽生えることがそのねらいとしてあるかについて、それぞれの作品について説明する。

\subsection*{Familiar / Strange(執筆中)}
後に出てくる「Relation」に比べると、\textit{grasp}の時間は短く、複雑度も低いため、この過程で体験者独自の試行が行われることは多くはない。しかし学内での展示に際して、徐々に変換が複雑になる中で、「ある形を作ってみようとする」ことや「途中から追いつけなくなった法則性を確かめる」ように手指を動かす方がいた。

\subsection*{Relation(執筆中)}
ボールという直接的に制御することができない対象とのあいだでのGraspを経験するので、\textit{grasp}の時間は長い。その過程で「左右に転がしてみる」、「投げ上げる」、「受け止める」といったことを試行する、前の作品よりも注意の向く対象が多い作品である。そして、このとき「自発的に芽生えた目的意識」こそが、制作者の意図を超えたものであると考える。

% \textit{grasp}には二つの性質がある。一つは、時間幅が注意を向けている対象によって、長い場合と短い場合があることである。目的意識が芽生えてからスムーズに操作できるようになる場合、'reach'から'manipulate'が近く、\textit{grasp}は短い、すなわち直感的で使いやすいものとして経験される。その一方、目的意識が芽生えてから試行錯誤を伴い、習熟に長い期間を要する場合、\textit{grasp}は長く、もどかしさを経験し、操れるようになった時に達成感を経験する。\\

% 二つ目に、\textit{grasp}の過程で他のことに対する意識が次々と芽生えることがある。具体的には「やってみるまでわからない」といった経験や、物事に対する解像度が高まる中で、当初とは異なる意識が芽生える状況に相当する。\\

% このコンセプトを展開し、「試行錯誤の余地」を設計することを目指したのが、修士作品《Grasp(er)》である。この作品では、\textit{grasp}を経験する中で、個人による創造的な活動が生まれ、\textit{grasp}という動作を行っているのではなく、そこに'er'の接尾辞がついた'Grasper'であると名付けた。