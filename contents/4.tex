\chapter{修士作品 Grasp(er)}
\section{概要}
ここまで、身体性の変容をその契機として「わからなさ」「ままならなさ」を経験することについてFelsの分類に基づいて整理を行い、前の\ref{graspについて}章で\textit{grasp}の定義を行った。

本章では、そうしたコンセプトに辿り着くまでに行ったプロトタイピングについて整理すると共に、作品の概要を述べる。また、本章の最後には、本作においてこの\textit{grasp}のモデルがどのように作品形態に実装されているかについて説明する。

\section{制作プロセス}
本節では、最終的な作品形態に至るまでのプロトタイピングについてと、その過程で構築されたライブラリの開発について説明する。
\subsection{プロトタイピング}
本作品の形態に至るまでに、総計60パターンのプロトタイプ制作を行なった。プロトタイピングにおける目的を時系列で3段階に大別すると、「コンセプトメイキング」、「変換表現によって錯覚する「手触り」についての発散」、そして「変換表現で異なる身体動作が引き出される要素の特定」である。
\subsubsection{コンセプトメイキング}
作品のもととなった試行は当初、「動きのスケッチツール」というまったく別の目的で作られたシステムである。\\
ピアノを弾く時のように、人は10本ある指を同時に操り、その組み合わせから柔軟で緻密な動作を作ることができる。手のストロークを記録し、静止画をスケッチする場として「紙とペン」があるように、手の運動を運動としてそのまま記録し、動きをスケッチする場として新しいインターフェースを設計する、という目的のもと、その変換方法や動きを入力するためのインターフェースを案じていた。その中で、手指の姿勢情報をもとにしながらも別の構造を動かすインターフェースを設計した。手指を手指の形のまま表現するのではなく、違う構造を操る中で、動きを作りながらアイデアが形作られていく体験が生まれるのではないかと仮説をたて、さまざまな変換方法のプロトタイピングを行なっていた。\\
しかし「変換された身体」を操る体験には、それを動かす体験自体に快があること、そしてこのようにして制作されたプロトタイプを展示した際、誰しもが同じような操作感を感じるわけではなく、「うまくでき、楽しいと感じる人」、「構造が理解できず、楽しいと感じられない人」とが現れたことに興味を持った。\\

\begin{figure}[H]
  \centering
  \includegraphics[width=15cm]{img/placeholder.png}
  \caption{IAMAS Open House 2022での展示のようす(2022年)}
  \label{fig:exhibit_2022}
\end{figure}

これをきっかけとして、手指の運動を用いた表現について「動きのスケッチツール」というテーマから離れ、「手指の変換表現」について幅広く検討する中で、こうした興味を掘り下げていくことにした。\\

\subsubsection{変換表現によって錯覚する「手触り」についての発散}
変換された手指を見ながら自分の身体を動かしていると、独特の気持ちよさや、ある動きをしようと力が入る感覚がある。物理的に何かに触れているわけではないが、何もないところで同じ動きをしていても知覚することのない、視覚情報と手指の運動の対応関係によって生じる身体感覚を、ここでは「手触り」と呼ぶ。
プロトタイピングの中で、変換表現の形状、マッピングの方法、推定された手指の姿勢情報の時間操作などを通して、この「手触り」にどのような幅が存在するのかについて探った。
\subsubsection*{形状}
形状については、指を動きの最小単位として、指ごとに独立したバリエーション、指同士を直列に繋ぎ合わせたバージョン、円形に繋ぎ合わせたバージョンなどを作成した。また、単位である指をどのように表現するかについては、「円形」、「くの字」、「ひょうたん型」などのバリエーションで表現を試みた。
\begin{figure}[H]
  \centering
  \includegraphics[width=15cm]{img/placeholder.png}
  \caption{ユニットのバリエーション}
  \label{fig:unit_valiation}
\end{figure}
\subsubsection*{マッピングの方法}
マッピングについては、1つの動きを複製して、1つの動きを複数のパーツの動きへと波及させるバリエーションなどを作成した。図\ref{fig:networked_finger}に示すプロトタイプでは、指先をクリックすると5本ある指のうちのいずれかの動きを追従する指が、指先に追加されるものである。どの指が付け加わるかはランダムで、指が新しく追加されるたびに、それがどの指の運動であるかを同定するには、一本一本指を動かして、どこがその指に対応しているのかについて同定する必要がある。
\begin{figure}[H]
  \centering
  \includegraphics[width=15cm]{img/placeholder.png}
  \caption{Networked Finger}
  \label{fig:networked_finger}
\end{figure}
\subsubsection*{手指の姿勢情報の時間操作}
最新の自分の動きだけを表示するのではなく、過去の動きも表示するプロトタイプにも取り組んだ。図\ref{fig:prototype_delay}に示すプロトタイプでは、5本の指の動きが等間隔に並べられているが、それぞれの指は、鉛直上向きの角度に現在の指の動き、そこから時計回りに、順次過去の指の動きが並べられている。このプロトタイプでは、指先を小さく動かすと、その動きが時計回りに伝播していくようすが見て取れる。これは例えばゼリー状の物体を触れた時のような、衝撃が物体全体へと伝播していくようすに見立てられるため、柔らかいオブジェクトに触れているときのような身体感覚を錯覚することがある。

\begin{figure}[H]
  \centering
  \includegraphics[width=15cm]{img/placeholder.png}
  \caption{過去の動きを用いた例}
  \label{fig:prototype_delay}
\end{figure}
\subsubsection{変換表現でIntimacyが引き出される表現への収束}
ここまで行なってきたプロトタイプから、以下の2つの観点から方向性を絞り、ブラッシュアップを試みた。
\subsubsection*{制御している感覚が強く引き出される表現}
1つ目の観点として、手指の微細かつ複雑な動きをもって、緻密な制御をしているという感覚が引き出される表現へと絞り込んでいくことにした。\\
その上で注目したのは、「関節を折り曲げる動き」に対する共感性である。\\
人間の手指は
その一方で、円の半径に関節の動きをマッピングさせるような表現については、縮んだり、膨らんだりする動きには共感しづらい。微細な運動が増幅される「気持ちよさ」があるが、この気持ちよさは身体動作との連動によってもたらされる気持ちよさではなく、動作に対するフィードバックに対する快、すなわちFelsのいう「Response」による快が強い。経験されるため採用しなかった。\\
さらに、1つの動きを複製して円形に配置したり、フラクタル的に配置するパターンを排除した。これについては、グラフィックデザイナーの女性が体験した際「構造としては緻密であるのに、動きは単純であるように感じる」と意見した。その理由として彼女は、「対称性が前面に出ているため、シンボルとしての印象を強く感じてしまう」「意識していないところも同時に動いている感覚があるため、自分の動きだと思えない」からではないかと推測した。

\subsection{ライブラリの開発}
本作品を構成するにあたり、基本的な関数をまとめたライブラリを開発した。ライブラリには、円滑に体験するための補完処理を実装している。
具体的には、ガウシアンフィルターによる平滑化処理、トラッキングが途切れた際の例外処理の2つである。

\subsubsection*{ガウシアンフィルターによる平滑化処理}
推定精度の問題から、モデルより推定される姿勢情報をそのまま出力すると、手指を動かしていなくても小刻みに振動したり、一時的なフレームレートの低下に起因してスムーズに動作していないように感じることがある。\\
そこで、体験者にフィードバックする際に使用する姿勢情報は、前後2フレーム分のフレーム情報にガウシアンフィルターを適用した平滑化処理を実装している。ただし、トラッキングが開始した直後は5フレーム分のフレーム情報を使用することができないため、この場合は取得できる限りのフレーム情報を用いて同じ処理を行なっている。そのため以下では、各フレーム情報に対する重みづけと、それを用いて体験者に提示される姿勢情報を求める上での一般式を示す。
モデルより推定された最新の姿勢情報を\(P_{n}\)、出力されている姿勢情報を\(S\)とすると、
  % 平滑化フレーム S の定義
  \begin{equation}
    S = \sum_{i=-2}^{2} w_i' \cdot P_{n+i}
    \end{equation}

ここで、\(w_i'\)は正規化されたガウシアンフィルタの重みを表す。正規化前の重み\(w_i\)は、
\begin{equation}
  w_i = \frac{1}{\sqrt{2\pi}\sigma} e^{-\frac{i^2}{2\sigma^2}}
  \end{equation}

正規化された重み\(w_i'\)は、
  % 重みの正規化
  \begin{equation}
  w_i' = \frac{w_i}{\sum_{j=-2}^{2} w_j}
  \end{equation}
と表現される。
この処理のため、最良時で60fps程度で取得される姿勢情報は、慢性的に0.3sほどの遅延を伴って体験者にフィードバックされることになる。

\subsubsection*{トラッキングが途切れた際の例外処理}
体験時、環境光や、手指を動かす範囲や速度の関係から、トラッキングが途切れることがある。素早い動きをしている最中に1フレームでも途切れると円滑に体験することができないため、この時は例外的に、トラッキングが途切れる直前のフレーム情報で失われたフレーム情報を埋め合わせる処理を実装した。また、トラッキングが途切れていることに起因する不快感は、本作品の体験外の問題なので、手指の動きがトラッキングできていない状態を視覚的にフィードバックするため、トラッキング不能時に塗りつぶしを透過する視覚効果を実装した(\ref{fig:opacity})。

\begin{figure}[H]
  \centering
  \includegraphics[width=15cm]{img/placeholder.png}
  \caption{トラッキング不能時の視覚効果}
  \label{fig:opacity}
\end{figure}

\section{作品構成}
一方で、プロトタイピングを通して同定した変換表現を通して生じた異なる身体像に対する共感覚のようなものは、変換された手を用いた作業に移行した瞬間に、注意が向きづらくなるといった課題が起こった。そこで、最終的な作品としてはその2つを分けて構成した。

\subsection*{Familiar / Strange}
「Familiar / Strange」は、手指の配置や関節の数・位置が次々と変化していく作品で、手の運動に対する再注目が起きることを目指している。作品は最初、手の形がそのまま現れた状態から開始し、指の並びや関節の配置の変化が起きる。一本一本の指がくの字の形をして積み上げられる様子をピークに、逆順に変化が巻き戻され、再び元の手の形に戻るという、3分10秒で1ループの構造となっている。
\begin{figure}[H]
  \centering
  \includegraphics[width=15cm]{img/placeholder.png}
  \caption{Familiar / Strange}
  \label{fig:familiar_strange}
\end{figure}

\subsection*{Relation}
「Relation」は、変化した手指を取り巻く関係が次々と変化していく作品で、手と直接的に制御できないボールの関係性におけるGraspに焦点を当てている。1つ目の作品と同様、最初は手が表示された状態から始まり、段階的に並び替えられた手指を覆う皮膜が現れ、ボールが現れる。最後は的が現れ、的を取るたびにボールの大きさが小さくなり、3つ連続して取ると、皮膜は消え、再び元の手の形に戻る。
トラッキングされた手指の位置が、手首から指ごとに分割され、左端から右へ、左手の小指から親指、そして右手の親指から小指の順に整列される。しばらくすると、指先以外の運動が捨象され、残された指先を結ぶ線が現れる。ここで、現れた線によって再び全ての指が1つのまとまりとして統合されることになるが、その線は後に現れるボールに対して衝突判定が適用される、新しい構造の手指を覆う皮膜のような機能を有する。皮膜のある領域を外れるとボールは下方に落下するが、そのあとは再び画面の中央にボールが出現する。
さらに一定時間が経過すると、皮膜の上方に白い点が現れる。白い点に対してボールを当てると、ボールは一回り小さくなり、ボールを落とさずに合計3回白い球に当てると、ボールは消失し、皮膜が現れるときと逆の順序を辿って画面の中の手は再びもとの形状に戻る。
\begin{figure}[H]
  \centering
  \includegraphics[width=15cm]{img/placeholder.png}
  \caption{Relation}
  \label{fig:relation}
\end{figure}

この作品は、1つめの「Familiar / Strange」と異なり、構造が変化した手指と、ボールという直接的に動かすことができない対象との関係性に注意が向けられることをねらいとした。