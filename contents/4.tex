\chapter{修士作品 Grasp(er)}
\section{概要}
\section{制作プロセス}
\subsection{プロトタイピング}
本作品の形態に至るまでに、総計60パターンのプロトタイプ制作を行なった。プロトタイピングにおける目的を時系列で3段階に大別すると、「コンセプトメイキング」、「」、そして「」である。
\subsubsection{コンセプトメイキング}
\subsection{ライブラリの開発}
\section{作品構成}
\subsection{Familiar / Strange}

\subsection{Relation}
そのままの手が、手首から指ごとに分割され、左端から右へ、左手の小指から親指、そして右手の親指から小指の順に整列される。しばらくすると、指先以外の運動が捨象され、残された指先を結ぶ線が現れる。ここで、現れた線によって再び全ての指が1つのまとまりとして統合されることになるが、その線は後に現れるボールに対して衝突判定が適用される、新しい構造の手指を覆う皮膜のような機能を有する。皮膜のある領域を外れるとボールは下方に落下するが、そのあとは再び画面の中央にボールが出現する。
さらに一定時間が経過すると、皮膜の上方に白い点が現れる。白い点に対してボールを当てると、ボールは一回り小さくなり、ボールを落とさずに合計3回白い球に当てると、ボールは消失し、皮膜が現れるときと逆の順序を辿って画面の中の手は再びもとの形状に戻る。

この作品は、1つめの「Familiar / Strange」と異なり、構造が変化した手指と、ボールという直接的に動かすことができない対象との関係性に注意が向けられることをねらいとした。
\section{本作品におけるモデルの適用}