\chapter{考察}

\section{今後の展望}
本節では、青木淳による「原っぱと遊園地」、ミゲル・シカールの遊び論について、ホイジンガとの比較から松永によって名付けられた「ふざけた遊び」との関係を整理し、今後の展望とする。

\subsection{「原っぱと遊園地」}
楽器は、表現したいことが滞りなく表現できるようになるためには、習得も必要で、触れて直ちに使いこなせるわけではないという点において、「使いづらい」と説明することもできる。しかし楽器は結果として「使いづらい」わけだが、意図して「使いづらい」体験を設計している、というわけではないだろう。そうではなく、道具を通して達成されること(楽器であれば、演奏するという用途)、人間の身体的特性を考慮した上での最適化された合理的な形状であって、「使いやすさ」を目指すわけにはいかない部分が多いということだ。建築家の青木淳は、
楽器は、決定ルールが「使いやすさ」を設計する態度から自由であるという点において、「原っぱ的自然」のプロダクトであると言える。

\subsection{シカールの「ふざけた遊び」}
こうした「原っぱ」的なプロダクトに対して
本研究が着目した\textit{grasp}のコンセプトは、