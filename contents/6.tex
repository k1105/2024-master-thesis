\chapter{考察}
インタビューを通して得られた結果について、本作で試みていたコンセプトと照らし合わせた上で、達成できたこと、できなかったことについて考察する。

\section{今後の展望}
本節では、青木淳による「原っぱと遊園地」、ミゲル・シカールの遊び論について、ホイジンガとの比較から松永によって名付けられた「ふざけた遊び」との関係を整理し、今後の展望とする。

\subsection{「原っぱと遊園地」}
楽器は、表現したいことが滞りなく表現できるようになるためには、習得も必要で、触れて直ちに使いこなせるわけではないという点において、「使いづらい」と説明することもできる。しかし楽器は結果として「使いづらい」わけだが、意図して「使いづらい」体験を設計している、というわけではないだろう。そうではなく、道具を通して達成されること(楽器であれば、演奏するという用途)、人間の身体的特性を考慮した上での最適化された合理的な形状であって、「使いやすさ」を目指すわけにはいかない部分が多いということだ。建築家の青木淳は、建築においてそこで行われることをあらかじめ決定されているような空間を「遊園地」とし、一方で行われることで空間が作られていくような空間を「原っぱ」と呼び、分類している。

\begin{quote}
  ともかく、廃校になった機能主義的小学校の空間は、ちょうど原っぱのように、人間にそれ に対するかかわり方の自由を与える。 原っぱとは、つまり空き地である。 宅地が造成され区画 される。これは人工的な営みである。 塀が築かれ、土地の形がきちんと確定される。一度は土地が均され雑草が刈り取られる。 そこまで行って、なにかの理由で、放置される。時間が経過して、 セイタカアワダチソウなどの雑草が、背丈ほど伸びてくる。そして、原っぱができ上がる。 更地というだけでは、原っぱではない。放置後の、適切な度合いの自然の遂行を必要とする。 
  廃校になった牛込原町小学校は、原っぱと同じく、人間の感覚とは一度は切れた決定ルールによって生成し、しかしその決定ルールが根拠を失った空間だったのである。そうして、偶然に、そこで人がつくることと、与えられる空間の規定力の対等が実現されていたのである。  
\end{quote}


楽器は、決定ルールが「使いやすさ」を設計する態度から自由であるという点において、「原っぱ的自然」を持つプロダクトであると言える。こうしたプロダクトに対して、メディアアーティストの久保田は「使いやすさ」を超越して「使いたさ」の重要性を説く。

\begin{quote}
  楽器は、リアルタイムに音を生成、コントロールするための道具である。そのインターフェイスの使いやすさが重要になるのはいうまでもない。 しかし、ピアノのインターフェイスは「はじめての人にも使えるように」あるいは「一目でわかるように」デザインされているだろうか?ギターは?トランペットやサックスの場合は?いずれの楽器も、 はじめての人にとっては音を出すだけでも四苦八苦の、使いにくい道具である。にもかかわらず、人はそんな使いにくい楽器に対して時間をかけてじっくりと向き合い、日々練習を重ね、少しずつスキルを向上させながら美しい調べを自在に奏でる夢を見る。そうした営みを支えているのは、人々の願いやビジョンである。(中略)「使いやすさ」のためにも、何よりもまず「使いたい」という願望が必要である。「どうやってやるか」ということよりも「何をやるか」のほうが先に来る。 豊富な機能を生かすも殺すもインターフェイス次第であることは間違いないが、その豊富な機能を使いたいと思えなければしょうがない。考えてみれば当然だ。 携帯電話のテンキーによる文字入力が良い例だ。何も工夫していないものが、結局は一番使いやすいインターフェイスになっている。重要なのは、表面的な改良ではなく、大きさや重さ、速度といった基本的な要目だ。妙な工夫をするよりもむしろ、シンプルであればあるほどスキルが活躍する余地が生まれる。ワンタッチで具体的なインターフェイスよりも、システマティックで抽象的なインターフェイスのほうが、多様な使用法とアウトプットを生み出すことができる。その名人芸と形容したくなるほどのスキルを生み出しているのは、人々の欲望だ。ここでも、欲望さえあれば、指が勝手に動いていく。技術革新の速度はそれなりに速いのかもしれないが、その気になった人間の適応力や柔軟性による変化の速度はもっと速い。人間の可能性は、底知れない。
\end{quote}

\subsection{シカールの「ふざけた遊び」}
こうした「原っぱ」的なプロダクトに対して
本研究が着目した\textit{grasp}のコンセプトは、