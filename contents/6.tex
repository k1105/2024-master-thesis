\chapter{考察}
\label{考察}
本章では、\ref{validation}章で説明した結果を踏まえ、それぞれの作品において、どのような\textit{grasp}が芽生えていたのか、そしてそれを通して、参加者はこの作品をどのように受け止めていたのかについて考察する。

\section{Familiar / Strange}
参加者1と2では、挙動を確認するための動き、すなわちResponseの動作が見られた。

さらに、体験の中でプリミティブな図像であるためか、何かしらの「見立て」をしていることが多いとわかった。参加者1は、指が縦方向の動きのみに拘束されてパタパタと上下する形になったときに、ピアノの演奏のような身体感覚を想起していた。また参加者4は、「頭の中で既視感を作って」体験していたと振り返る。こうした「見立て」が生じることは、体験する人がそこに目的意識を見出して、対象を捉えようとすることの表れと考えられる。

何かしら参照できる他の身体動作や、見立てが可能であることが、体験の中で目的意識や注意を自分で作って体験することにつながるのではないかと語っている。
一方で参加者3は、手でフレームを作るような動きをしており、それに対して画面の中の手は思うように形ができず、「途中から飽きていた」と語っていた。
指一本一本の単位で手指の構造が切り替わっていく構成ではあったが、そうした画面の変化について「認知が追いつかなかった」と振り返り、手指を使って形を作っていたというイメージに対して、画面の中の振る舞いがそれとは全く異なることについて注意が向かなかったのではないかと考える。
また、「楽しみを探れた」という意見もあった。
\section{Relation}
興味深いのは、体験におけるもどかしさについてはトラッキングの精度とはあまり関係していないという点である。

参加者1は、。また参加者3は、
これは、初めて体験する人にとって、トラッキングが途切れていることがもどかしさとして経験されるほど、Intimacyは高まっていないということが示唆される。
参加者1は、

インタビューを通して得られた結果について、本作で試みていたコンセプトと照らし合わせた上で、達成できたこと、できなかったことについて考察する。

シビアな身体動作が求められるような目的意識と、そうでないものがあるということがわかった。「ボールを的に当てる」といった、明確な正解が存在する動きについては、指のトラッキング精度が低下していることに対して不快感を訴える意見があった一方で、「Familiar / Strange」ではトラッキングができていなくてももどかしいと感じるような意見は少なかった。
体験者1は、「気持ちよさを感じるポイントは、ゲームを進めていく中で変化してったけど、最初は「球を上げる」っていう目標になって、それを達成する気持ちよさを超えて、次の目標が見つかったときには、的に当てないと気持ちよくない」と振り返った。

さらに、「思い通りに動く」とは「連動性が担保されていること」ではないか、という仮説が新たに生じた。「思い通り」という言葉は、相手に命令する立場からすると、期待している結果と実際の挙動が一致している際に生じるが、「control」の状態下での「思い通り」とは、参加者4が体験を振り返るコメントにもあるように、「トラッキングができて」いて、「時差を極力なくし」た動きが「自分の思い通りの動きになってくれる状態」、すなわち「連動性」ではないかということが示唆された。その上で、「Belonging:帰属感」のようなIntimacyが生じるためには、その動きがさらに細かく制御できていることが重要ではないかと考えられる。

\section{今後の展望}
本節では、今回の研究では言及することができなかったが、今後このコンセプトをより発展させていくにあたり、論じていく必要があると考える2つの論点について整理する。1つは、青木淳による「原っぱと遊園地」、もうひとつは、ミゲル・シカールの遊び論について、ホイジンガとの比較から松永によって名付けられた「ふざけた遊び」に関する議論である。

\subsection{「原っぱと遊園地」}
楽器は、表現したいことが滞りなく表現できるようになるためには、習得も必要で、触れて直ちに使いこなせるわけではないという点において、「使いにくい」と説明することもできる。しかし楽器は結果として「使いにくい」わけだが、意図して「使いにくい」体験を設計している、というわけではないだろう。そうではなく、道具を通して達成されること(楽器であれば、演奏するという用途)、人間の身体的特性を考慮した上での最適化された合理的な形状であって、「使いやすさ」を目指すわけにはいかない部分が多いということだ。建築家の青木淳は、建築においてそこで行われることをあらかじめ決定されているような空間を「遊園地」とし、一方で行われることで空間が作られていくような空間を「原っぱ」と呼び、分類している。

\begin{quote}
  ともかく、廃校になった機能主義的小学校の空間は、ちょうど原っぱのように、人間にそれ に対するかかわり方の自由を与える。 原っぱとは、つまり空き地である。 宅地が造成され区画 される。これは人工的な営みである。 塀が築かれ、土地の形がきちんと確定される。一度は土地が均され雑草が刈り取られる。 そこまで行って、なにかの理由で、放置される。時間が経過して、 セイタカアワダチソウなどの雑草が、背丈ほど伸びてくる。そして、原っぱができ上がる。 更地というだけでは、原っぱではない。放置後の、適切な度合いの自然の遂行を必要とする。 
  廃校になった牛込原町小学校は、原っぱと同じく、人間の感覚とは一度は切れた決定ルールによって生成し、しかしその決定ルールが根拠を失った空間だったのである。そうして、偶然に、そこで人がつくることと、与えられる空間の規定力の対等が実現されていたのである。  
\end{quote}


楽器は、決定ルールが「使いやすさ」を設計する態度から自由であるという点において、「原っぱ的自然」を持つプロダクトであると言える。こうしたプロダクトに対して、メディアアーティストの久保田は「使いやすさ」を超越して「使いたさ」の重要性を説く。

\begin{quote}
  楽器は、リアルタイムに音を生成、コントロールするための道具である。そのインターフェイスの使いやすさが重要になるのはいうまでもない。 しかし、ピアノのインターフェイスは「はじめての人にも使えるように」あるいは「一目でわかるように」デザインされているだろうか?ギターは?トランペットやサックスの場合は?いずれの楽器も、 はじめての人にとっては音を出すだけでも四苦八苦の、使いにくい道具である。にもかかわらず、人はそんな使いにくい楽器に対して時間をかけてじっくりと向き合い、日々練習を重ね、少しずつスキルを向上させながら美しい調べを自在に奏でる夢を見る。そうした営みを支えているのは、人々の願いやビジョンである。(中略)「使いやすさ」のためにも、何よりもまず「使いたい」という願望が必要である。「どうやってやるか」ということよりも「何をやるか」のほうが先に来る。 豊富な機能を生かすも殺すもインターフェイス次第であることは間違いないが、その豊富な機能を使いたいと思えなければしょうがない。考えてみれば当然だ。 携帯電話のテンキーによる文字入力が良い例だ。何も工夫していないものが、結局は一番使いやすいインターフェイスになっている。重要なのは、表面的な改良ではなく、大きさや重さ、速度といった基本的な要目だ。妙な工夫をするよりもむしろ、シンプルであればあるほどスキルが活躍する余地が生まれる。ワンタッチで具体的なインターフェイスよりも、システマティックで抽象的なインターフェイスのほうが、多様な使用法とアウトプットを生み出すことができる。その名人芸と形容したくなるほどのスキルを生み出しているのは、人々の欲望だ。ここでも、欲望さえあれば、指が勝手に動いていく。技術革新の速度はそれなりに速いのかもしれないが、その気になった人間の適応力や柔軟性による変化の速度はもっと速い。人間の可能性は、底知れない。
\end{quote}

\subsection{シカールの「ふざけた遊び」}
こうした「原っぱ」的なプロダクトに対して、本研究が期待するような関わり方について述べるのは、ミゲル・シカールによる遊び論で語られる、「ふざけた遊び」の性質を持った遊び心の発露であろう。\\
遊びは存在のためのポータブルな道具である。ゲームに代表されるような遊びの「形式」よりも、「世界のうちに存在するモードの一種」としての遊び、つまり人間が人間としてあるあり方のひとつとしての遊びである。遊びは文脈に依存する。これは遊びが、物や環境やテクノロジーや人などからなる、その都度の「文脈」を使うかたちで生じるということだ。シカールの考えでは、もともと遊びのためにデザインされたものであるはずのゲームのルールですら、その本来の目的を無視して流用してしまうのが、遊び心という態度であり、遊びという「人間存在のモード」なのだ。\\
本研究が着目した\textit{grasp}のコンセプトは、こうした人間の感情の発露を期待する。その意味で、誰もが同じようにそれに価値を見出すような、一般的な利便性よりも、属人的な楽しさを志向したアプローチであると言えるだろう。