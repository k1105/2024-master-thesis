\chapter{\textit{Grasp}モデルについて}
\label{define_model}

\section{モデルの概要}
「grasp」は「把握」を意味する動詞・名詞だが、日本語でも物体を把持する手の動作であると同時に、概念について理解するときにも用いる。ここでの\textit{grasp}とは、「人がある対象に注意や目的意識を抱き、注意を向けた対象を確かめるため、または目的の達成に向けて試行を行う期間」のことを指す。手指の構造を変化させる本作は、このコンセプトに注目し、手指の変換をきっかけに起こる注意や、変換した手指とボールとの関係のもとで生じる目的意識が、体験者の中で自発的に生まれることを目指した。
本研究では、作品体験についてのインタビューを通してそれが実現していたことを示し、そのデザインプロセスと最終的な作品形態についての考察を通して、\textit{grasp}の設計に向けた指針となることを目指す。
本モデルは、Felsによる身体性のカテゴリーの分類に基づき

\section{既往研究との比較}
\subsection{Felsによる身体性のカテゴリー}
\textbf{Response:}\\
人はオブジェクト(例えば、コンピューターやアート作品)に対する働きかけを行い、その反応から感情的な反応や理解を得る状態を指す。この時点では、オブジェクトは人と分かれたものとして知覚されている。

\textbf{Control:}\\
人がオブジェクト(例えば、楽器やペイントブラシ)を自分自身の延長として使用し、その操作によって感情的な満足や美的体験を得る状態を指す。Felsは、自身の作品であるIamascopeにおいて、人の身体動作にシステムが素早く視覚的なフィードバックを返すことから、こうした関係性が生じると説明する。「自分自身の延長」として経験される感覚について言及しており、またそれが追従性の高いグラフィックによってもたらされるという記述は、渡邊がマウスカーソルやスマートフォンに対して用いた「操作時の指とグラフィックの追従性が高い」インターフェースという説明と同等のものである。このことからFelsのいうControlとは、Gallagherの「sense of ownership」と同じものを指していると考えられる。

\textbf{Contemplation:}\\
人がオブジェクトに対する働きかけはせず、人がそのオブジェクトからの信号やメッセージを内省や反映を通じて体験する状態を指す。具体例として、絵画の鑑賞体験が挙げられている。

\textbf{Belonging:}\\
オブジェクトによって人が動かされているような経験を指す。人はそのオブジェクトによって提供される体験を通じて感情的な反応を得る。ここでは、オブジェクトが人の体験や感情を形作る役割を果たす。
