\chapter{\textit{Grasp}について}
\label{define_model}

\section{概要}
\textit{grasp}は「把握」を意味する動詞・名詞だが、ここでの\textit{grasp}とは「人がある対象に注意や目的意識を抱き、注意を向けた対象を確かめるため、または目的の達成に向けて試行を行う期間」とする。手指の構造を変化させる本作は、このコンセプトに注目し、手指の構造の変換をきっかけに起こる注意や、変換した手指とボールとの関係のもとで生じる目的意識が、体験者の中で自発的に生まれることを目指した。本章では、この\textit{grasp}についての詳細を示す。

\textit{grasp}には二つの性質がある。一つは、時間幅が注意を向けている対象によって、長い場合と短い場合があることである。目的意識が芽生えてからスムーズに操作できるようになる場合、'reach'から'manipulate'が近く、\textit{grasp}は短い、すなわち直感的で使いやすいものとして経験される。その一方、目的意識が芽生えてから試行錯誤を伴い、習熟に長い期間を要する場合、\textit{grasp}は長く、もどかしさを経験し、操れるようになった時に達成感を経験する。

二つ目に、\textit{grasp}の過程で他のことに対する意識が次々と芽生えることがある。具体的には「やってみるまでわからない」といった経験や、物事に対する解像度が高まる中で、当初とは異なる意識が芽生える状況に相当する。

このコンセプトを展開し、「試行錯誤の余地」を設計することを目指したのが、修士作品「Grasp(er)」である。この作品では、\textit{grasp}を経験する中で、個人による創造的な活動が生まれ、\textit{grasp}という動作を行っているのではなく、そこに'er'の接尾辞がついた'Grasper'であると名付けた。

'Grasper'は、「Familiar / Strange」と「Relation」という二つから構成された作品群である。これらに共通する目的は、\textit{grasp}の中で個人が目的意識や興味を抱くことで、\textit{grasp}が連鎖的に生じることである。制作者は「このようなことをしてほしい」という行為の中身を設計せず、体験する個人がその中で次々と注目する対象を見出すことによって、個人が行為を創造していくことを目指している。

このため、「手指の構造や、手指を取り巻く環境を変化させることで、手指の運動に注目する構造」を作ることに取り組んだ。このときの「注目」が起点となって\textit{grasp}の期間が生じるが、その先々で起こる体験は個人に委ねられ、明示的な目的は設定されていない。

この作品群の筐体の設計は、画面の中に出力される図像が「自分の手指に替わる存在」として認識されるように意図されている。体験中に意識が集中できるような構成が目指され、手指を動かす際にはできるだけ自由に動かせるようになっている。
