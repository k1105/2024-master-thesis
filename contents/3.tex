\chapter{\textit{Grasp}モデルについて}

\section{モデルの概要}
「grasp」は「把握」を意味する動詞・名詞だが、日本語でも物体を把持する手の動作であると同時に、概念について理解するときにも用いる。ここでの\textit{grasp}とは、「人がある対象に注意や目的意識を抱き、注意を向けた対象を確かめるため、または目的の達成に向けて試行を行う期間」のことを指す。手指の構造を変化させる本作は、このコンセプトに注目し、手指の変換をきっかけに起こる注意や、変換した手指とボールとの関係のもとで生じる目的意識が、体験者の中で自発的に生まれることを目指した。
本研究では、作品体験についてのインタビューを通してそれが実現していたことを示し、そのデザインプロセスと最終的な作品形態についての考察を通して、\textit{grasp}の設計に向けた指針となることを目指す。
本モデルは、Felsによる身体性のカテゴリーの分類に基づき

\section{既往研究との比較}
\subsection{Felsによる身体性のカテゴリー}
\textbf{Response}: a case where the person perceives the object to be separate from their own self. 

\textbf{Control}: the person feels like the object is an extension of their self. Fels describes this intimate
relation as occurring quite quickly within Iamascope due to the system’s prompt visual feedback [8].

\textbf{Contemplation}: not an interactive one. In this relationship the person sees their self as separate from the object. this relation as similar to that between a painting and its viewer.

\textbf{Belonging}: the person feels like the object is controlling them and they get pleasure from being thus controlled.