\chapter{まとめ}
\label{matome}
本研究の目的は、人と道具や機械における「人馬一体」のような一体感をデザインするための指針や具現化するための方法を明らかにすることである。

本研究では、手指の変換表現からアプローチした。探索的におこなったプロトタイピングの中から、その生起に関わると考えられる要素について検討した。そして、その要素から構成された修士作品《Grasp(er)》を制作した。随所に「人馬一体感」が生起することをねらいとして構成されたこの作品の体験を、「``一体化''としてのembodiment」から人と対象との関係を分類したとみられるFelsの概念を用いて捉えることで、それが何を達成するデザインであったのかについての仮説を立てた。4名の作品体験を振り返ることから分析を行った。この作品を初めて体験する4名を対象に、体験中に考えていたことをVideo Cued Recall手法により想起してもらい、体験中の様子を記録した動画と合わせて分析した。その結果、多くの場面において、体験の中で手指の動かし方が作品の構成に応じて変化することを通して、一体感が向上していったことが確認できた。しかしながら、「見立て」がうまくできないと一体感が向上しない、体験中に目標が変化したことで一体感が喪失する、といったことが起きていたことも分かった。これらの結果から、一体感が向上するにはFelsが分類した4つのうち、2つが同時に起きる期間があることが重要であるということが示唆された。それを「grasp」と名付け、Felsの分類を拡張することを提案した。

\section{今後の展望}

本節では今後に向けた展望として、体験の中での「目的意識」についてどのような方向性からアプローチするべきかについて述べる。
% この概念をより発展させていくにあたり、論じていく必要があると考える2つの論点について整理する。1つは、青木淳による「原っぱと遊園地」、もうひとつは、ミゲル・シカールの遊び論に関する議論である。

建築家の青木淳\cite{aoki2004harappa}は、建築においてそこで行われることをあらかじめ決定されているような空間を「遊園地」とし、一方で行われることで空間が作られていくような空間を「原っぱ」と呼び、分類している。本作品における「目的意識」の位置付けはあくまで明示的なものではなく、体験の複雑度がもたらす「創発」であった。その境界を捉えるにあたって、青木の「原っぱ」と「遊園地」の区分は示唆的である。これは建築という空間設計の話だけではなく、人馬一体感の生起を目指したプロダクトや体験を設計するための、目的意識が創発するための余地を考えるにあたり、重要ととらえている。

% \begin{quote}
%   ともかく、廃校になった機能主義的小学校の空間は、ちょうど原っぱのように、人間にそれ に対するかかわり方の自由を与える。 原っぱとは、つまり空き地である。 宅地が造成され区画 される。これは人工的な営みである。 塀が築かれ、土地の形がきちんと確定される。一度は土地が均され雑草が刈り取られる。 そこまで行って、なにかの理由で、放置される。時間が経過して、 セイタカアワダチソウなどの雑草が、背丈ほど伸びてくる。そして、原っぱができ上がる。 更地というだけでは、原っぱではない。放置後の、適切な度合いの自然の遂行を必要とする。 
%   廃校になった牛込原町小学校は、原っぱと同じく、人間の感覚とは一度は切れた決定ルールによって生成し、しかしその決定ルールが根拠を失った空間だったのである。そうして、偶然に、そこで人がつくることと、与えられる空間の規定力の対等が実現されていたのである。  
% \end{quote}

しかし余地それ自体は何も生まないのであって、その上でいかにして強い目的意識が芽生えるのかについては、「遊び」心の発露について捉える必要がある。今後は、これらの議論と関連付けながら「人馬一体感」の生起における目的意識について、さらなる検討を深めていくことを考えている。

% に対して、本研究が期待するような関わり方が生起するためには、メディアアーティストの久保田が「使いやすさ」を志向することではなく、「使いたさ」を志向することの重要性を説くように、「強い目的意識」が生じることが重要である。

% \begin{quote}
%   楽器は、リアルタイムに音を生成、コントロールするための道具である。そのインターフェイスの使いやすさが重要になるのはいうまでもない。 しかし、ピアノのインターフェイスは「はじめての人にも使えるように」あるいは「一目でわかるように」デザインされているだろうか?ギターは?トランペットやサックスの場合は?いずれの楽器も、 はじめての人にとっては音を出すだけでも四苦八苦の、使いにくい道具である。にもかかわらず、人はそんな使いにくい楽器に対して時間をかけてじっくりと向き合い、日々練習を重ね、少しずつスキルを向上させながら美しい調べを自在に奏でる夢を見る。そうした営みを支えているのは、人々の願いやビジョンである。(中略)「使いやすさ」のためにも、何よりもまず「使いたい」という願望が必要である。「どうやってやるか」ということよりも「何をやるか」のほうが先に来る。 豊富な機能を生かすも殺すもインターフェイス次第であることは間違いないが、その豊富な機能を使いたいと思えなければしょうがない。考えてみれば当然だ。 携帯電話のテンキーによる文字入力が良い例だ。何も工夫していないものが、結局は一番使いやすいインターフェイスになっている。重要なのは、表面的な改良ではなく、大きさや重さ、速度といった基本的な要目だ。妙な工夫をするよりもむしろ、シンプルであればあるほどスキルが活躍する余地が生まれる。ワンタッチで具体的なインターフェイスよりも、システマティックで抽象的なインターフェイスのほうが、多様な使用法とアウトプットを生み出すことができる。その名人芸と形容したくなるほどのスキルを生み出しているのは、人々の欲望だ。ここでも、欲望さえあれば、指が勝手に動いていく。技術革新の速度はそれなりに速いのかもしれないが、その気になった人間の適応力や柔軟性による変化の速度はもっと速い。人間の可能性は、底知れない。(久保田晃弘, 「遙かなる他者のためのデザイン」)
% \end{quote}

% しかし、こうした目的意識の生起については、自発的に発生するものもあれば、コミュニティの単位などで創発する場合もあるのではないかと考えられる。

% こうした創発について述べるのは、ミゲル・シカールによる遊び論で語られる、「ふざけた遊び」の性質を持った遊び心の発露であろう。

% \begin{quote}
%   遊びは存在のためのポータブルな道具である。ゲームに代表されるような遊びの「形式」よりも、「世界のうちに存在するモードの一種」としての遊び、つまり人間が人間としてあるあり方のひとつとしての遊びである。遊びは文脈に依存する。これは遊びが、物や環境やテクノロジーや人などからなる、その都度の「文脈」を使うかたちで生じるということだ。シカールの考えでは、もともと遊びのためにデザインされたものであるはずのゲームのルールですら、その本来の目的を無視して流用してしまうのが、遊び心という態度であり、遊びという「人間存在のモード」なのだ。  
% \end{quote}

% 本研究が着目した「人馬一体感」は、こうした人間の感情の発露を期待する。その意味で、誰もが同じようにそれに価値を見出すような、一般的な利便性よりも、属人的な楽しさを志向したアプローチであると言えるだろう。

