\documentclass[dvipdfmx]{jreport}
% \usepackage{graphicx}
\usepackage[dvipdfmx]{graphicx}
\usepackage{sty/masterthesis}
%\usepackage{showkeys}%推敲用
\usepackage{lscape}	%横
\usepackage{mathptmx} % Timesへ
\usepackage{comment}
%\usepackage{booktabs}  %表結合
%\usepackage{multirow}  %表結合
%\usepackage{sty/slashbox}   %斜線
\usepackage{dcolumn}
\usepackage{array}
\usepackage{here}
\usepackage{enumerate}
\pagestyle{plain}
\makeatletter
\def\@cite#1{$\m@th^{\hbox{\@ove@rcfont #1)}}$} 
\def\@biblabel#1{[#1]}                              %参考文献[]へ
\def\@cite#1{{\mbox{\scriptsize{[#1}]}}}            %参考文献[]を文右へ
\makeatother
\setcounter{tocdepth}{2}                            % サブセクションも目次に出力
\begin{document}

% %----------------------------------------------------------------------
% % 表紙(年度,邦文題目,英文題目,指導教員,氏名,学籍番号を記入)
% % ※題目には適宜改行(\\)を手動で入れる
% %----------------------------------------------------------------------
% \年度{2021}
% \邦文題目{歯科インプラント治療における拡張現実を用いたサージカルガイドの開発と評価}
% \英文題目{Development and its evaluation of a Surgical Guide Using Augmented Reality for Dental Implantation}
% \指導教員{赤倉 貴子 教授,加納 徹 助教}
% \氏名{山岸 奏大 白井 清貴}
% \学籍番号{4617087,4618037}
% \表紙出力

% %----------------------------------------------------------------------
% % 要旨
% %----------------------------------------------------------------------
% \要旨{既存の歯科インプラント治療に用いられるプレート式サージカルガイドには,製造コストがかかる,一部の患者や状況によっては使用できないといった課題があげられている.本研究では,プレート式サージカルガイドの欠点を解決するシステムとして,歯列とコントラアングルハンドピースのマルチモデルトラッキングによるオクルージョン処理,ならびにドリリング位置や角度を通知する機能を備えたAR式のサージカルガイドシステムの開発を行なった.実習用模型を対象としたドリリングを実施しその偏差の平均を評価した結果,尖部で1.654 mm, 頸部で 2.000 mm,角度で2.954°となり,従来のプレート式サージカルガイドに近い精度でのインプラント埋入が可能な手法であることが示唆された.歯列とコントラアングルハンドピースのモデルトラッキングの精度や安定性が向上すれば,臨床においても活用できる.}
% \英文要旨{Generally, the surgical templates are used in existing dental implant positioning. However, they have some problems such as high manufacturing costs and inability to be used in some patients and situations. In this study, to solve the disadvantages of the surgical templates, we developed an occlusion-aware surgical guide system using Augmented Reality(AR). Occlusion is achieved using multi-model tracking based on three-dimensional geometry of dentition and contra-angle handpiece. Also, this system notifies the operator of the drilling position and angle. As a result of the drilling performance evaluation, the mean deviations between preoperatively planned and postoperative implants were 1.654 mm at the apex, 2.000 mm at the entry point, and 2.954° at the angle. This result suggests that our proposed system is capable of placing implants with accuracy similar to that of conventional surgical templates. When the accuracy and stability of model tracking of dentition and contra-angle handpiece are improved, this method can be used in clinical practice.}
% \要旨出力


%----------------------------------------------------------------------
% 目次
%----------------------------------------------------------------------
\contents

% 付図・付表のリストを作成する
\appendixpage{1}

%---------------------------------------------------------------------
% 1章 序論
%----------------------------------------------------------------------
\chapter{はじめに}
\label{introduction}

\section{研究背景}
\label{subject}

産業革命、コンピュータによる情報革命を経て、人々を取り巻くものに機械や情報処理を含まれ高度化していく中で、ヒューマンインターフェース設計が重要となった。
ここで目指された人々と機械との最適な関係とは、「より使いやすく有用に」\footnote{国際標準化機構(ISO)による「人間中心設計」の導入部\cite{hcd}より}することとされ、エルゴノミクス、人間中心設計といった方法論がそれを具現化してきた。
この「使いやすさ」を、道具の使用における行為と結果の関係から捉えると、それは原始的な道具のように直接的であり、道具そのもに対する意識がなくなっていくような「透明化」こそが、ヒューマンインターフェースにおける理想だとされてきた。

渡邊は、その理想を実現する指針として「自己帰属感」という概念を導入し、例えばマウスカーソルやスマートフォンのような「操作時の指とグラフィックの追従性が高い」インターフェースは自身の一部や延長として感じられる、「透明」なインターフェースであると説明する。

確かにこうしたインターフェースを設計していくことは、複雑で高度な道具の力を借りて人間の活動の可能性を拡げることに貢献してきた。しかし「透明」なインターフェースを目指す一方で、人々がある目的を達成できるようになるまでの「困難」と、それを克服して自分なりのやり方を創造していくような、高度な人間の能力の発揮と喜びを感じる機会については、これまで見落とされてきたのではないだろうか。

例えば、ピアノを初めて弾くとき、奏者は手の大きさによる制約を受け、最初のうちは左右別々に指を動かすだけでも困難さを経験する。しかし、試行錯誤を経て、ピアノの制約と自身の身体能力との間で折り合いをつけていくことで、ようやく楽器を通して表現ができるようになる。ミュージシャンのスキャットマン・ジョン(ジョン・ポール・ラーキン)は、「吃音症」という発話障害を抱えながらも、「自身の身体」という切っても切り離せない存在をコントロールできない中、むしろその症状を逆手に取るように「テクノスキャット」という独自の歌唱法を開拓した。こうした思い通りにいかなさ、すなわち他者性と向き合う中で、自分なりの扱い方を見出していく過程とは、創造的で喜びのある、いわゆる「人馬一体」と言われるような関係性ではないか。

こうした関係を設計するとしたら、「それはどのようなものであるか」を深く理解し、その上で「いかに設計できるのか」という問いに取り組む必要がある。

本研究では、手指の異なる形状への変換から「身体の中の他者性」を経験させる表現を通してこの問いについて探索し、修士作品《Grasp(er)》を制作した。さらに、その体験を説明するものとして\textit{grasp}という概念を定義した。
本研究の狙いは、\textit{grasp}の観点から人と道具、機械、あるいは人体の中にある他者性との関係における「困難さ」を捉えることで、これまで見落とされてきた関係性について説明し、またそうした関係性はどのようにして引き出すことができるのかを提示することである。

\section{リサーチクエスチョン}
\label{research_question}
前節では、研究背景に触れ、本研究が「人馬一体」と言われるような、相互の折り合いをつけながら生まれる親密さについて探索するものであると説明した。ここでは、探索の上での中心的な問いについて明記する。

\begin{quote}
\textbf{対象からの影響も受けつつ、相互の折り合いをつけながら生まれる一体感はどのように生まれ、そうした関係性が芽生える状況とはいかに設計できるのか。}
\end{quote}

本研究では、「手指の異なる形状への変換」を通して「身体の中の他者性」を経験させる表現を探索することからこの問いに迫った。

\section{「身体の中の他者性」に取り組む動機}
\label{prototyping_concept_making}
本研究は、\ref{research_question}節で示した問いに、手指の異なる形状への変換から「身体の中の他者性」を経験させる表現を通してその答えや詳細な説明についての探索を試みるものである。この観点から取り組むこととしたのは、もともと抱いていた\ref{subject}節のような問題意識に対して、当初はその問題意識とは無関係に進めていた「動きのスケッチツール」という目的の習作「Digitize」を展示した際の体験者の様子が、その切り口になると捉えたことがきっかけとなったためである。
この節では、そうした着想に至るまでの経緯について説明することで、手指の異なる形状への変換から「身体の中の他者性」を経験させる表現を通してこの問いへアプローチすることの動機とする。

ここで制作した「Digitize」とは、手指の動きを別の形へとマッピングさせた3つの変換表現から構成された作品である。

静止画について構想を膨らませる際、「紙とペン」を通して直接イメージをスケッチできるが、動きについてはそれに相当するほど、直感的かつ、高い表現力でスケッチができるツールが見当たらなかった。そこで、動きに関して高い表現力を有する手指の動きを使って、「動きをスケッチする」ツールを設計することを案じていた。

しかし手指は人間の身体の中でもとりわけ随意に、高い表現力で動かすことのできる器官である。もし「手ではない形」を通してでも、様々な形や動きを自由に作れるのなら、人間の身体構造による制約を超えた「動きのスケッチ」ができるのではないか。

そうして、手指の動きをトラッキングしつつ、画面上でその動きは別の形へマッピングされて動くインターフェースについて、プロトタイピングを行い、その過程をIAMAS Open House 2022にて展示した\footnote{\url{https://k1105.github.io/eee_openhouse_2022/}}。この時点では、動きを記録する機能は存在しておらず、あくまで手と、それに伴って動く手指とは異なる形の動きが確認できるだけのものであった。

\begin{figure}[H]
  \centering
  \includegraphics[width=15cm]{img/openhouse2022.jpeg}
  \caption{IAMAS Open House 2022での展示の様子(2022年)}
  \label{fig:exhibit_2022}
\end{figure}

\begin{figure}[H]
  \centering
  \includegraphics[width=15cm]{img/openhouse2022_interface.png}
  \caption{Digitizeのインターフェース(2022年)}
  \label{fig:exhibit_2022_interface}
\end{figure}

しかし展示を行うと、この試作には当初想定していなかった魅力があることに気づいた。それは、指の動きが単に、別の構造にマッピングされただけであるのに、別の構造の手を動かす体験はそれだけで興味を惹くものだということである。手指の異なる形状への変換が3パターン展示された状態のこの展示で、10分以上興味を持って体験する方が複数名いた。また、制作されたプロトタイプを展示した際、指先を動かすだけでなく、カメラに対して手を近づけたり遠ざけたり、手を裏返したりするなど、さまざまな体験の方法が現れた。

これは、「手指の動きに反応している」ということは分かっていても「どのように対応しているのか」がはっきりせず、それを確かめるように身体を動かしていると考えた。

仕組みを知っている制作者にとっては自明なことだが、自己の運動とどう対応するのかを知り得ない体験者は、自身の体を動かして観察されたことを通して推測することになる。それゆえに、この仕組みを通して思い描く身体像に、体験者ごとに違いがあるのではないか。

こうした体験者ごとの違いは、「わかりやすい」ものを対象としたときよりも、「わかりにくい」ものを対象としたときの方が顕著に現れると考えた。その上で、わかったようで分からない、行きつ戻りつな感覚に陥りながらも「わからなかった」と一蹴されず、それを分かろうとして向き合う様子が続いたことに、先の問題意識に応えるものがあるのではないかと考えた。


\section{研究概要}
本研究では、人間と道具や機械との関係において、人にとって道具や機械が「より使いやすく、有用に」あるという関係ではない、対象の他者性と向き合い、折り合いをつけていく経験を通して生まれる「一体感」を目指し、それがどのように生まれるのかについて探求する。この意味での「一体感」とは例えば、ピアノの演奏やバイクの運転など、人間が道具や機械の特性を理解し、使いこなすための能力を身につけることで生じる人間と対象との関係である。

この主題に、手指の異なる形状への変換を通して「身体の中の他者性」を経験させる表現を探索することから迫り、修士作品《Grasp(er)》を制作した。
また、人間と対象との関係について「Embodiment:一体化」の観点から分類したSydney Felsの4つのカテゴリや「Intimacy:親密さ」の概念に基づき、本研究では\textit{grasp}という概念を提案した。この概念を用いて本作から「一体感」が生じるまでの過程についての仮説を立て、4名の作品体験を振り返ることから、それが実現しているかについて考察する。

\section{本研究の目的}
本研究の目的は、対象の他者性と向き合い、折り合いをつけていく経験を通して生まれる「一体感」について、\textit{grasp}の観点から人と道具、機械、あるいは人体の中にある他者性との関係における「困難さ」を捉えることで、これまで見落とされてきた関係性について説明し、またそうした関係性はどのようにして引き出すことができるのかを提示することである。

\section{本論文の構成}
本章では、研究背景を問題提起の形で示し、その上でリサーチクエスチョンを提示する。また、その問いを探索する切り口として本研究が着目した「手指の異なる形状への変換」について、その観点から取り組む動機と、研究の概要を示した。

第\ref{related_works}章では、本研究の取り組みを説明するにあたり、重要な分類を行なったSydney Felsの人間と対象との関係性についての議論を紹介し、本研究の関心をより詳細に示す。その上で、これまでにどのような実践があったのか、先行事例について紹介し、本研究がどのような貢献となるのかについて示す。

第\ref{graspについて}章では、本研究が提示する\textit{grasp}について詳細に説明する。その上で、Felsの分類との関係性と、この概念を用いることから修士作品の体験についてのねらいを説明する。また、この概念に至るまでのプロトタイピングの過程を示すことで、この概念を主張する根拠とする。

第\ref{about_grasper}章では、作品概要と、そこに至るまでのプロトタイピングの分析を通して、作品形態について説明する。

第\ref{validation}章では、そのようなねらいのもと制作された本作品が、実際どのように経験されるのかについての質的調査を行うため、Video Cued Recallという手法を用いて体験について振り返り、実際の体験がどのようなものであったのかを踏まえて、モデルとの関係性について考察する。

第\ref{考察}章では、行った調査の結果からどのような体験の作品であったかを振り返り、本作品が狙いとしていたこと、あるいはその範疇を超えて、作品として何が達成されたのかについて考察する。

第7章では、これまでの議論をまとめ、最後に今後の展望として、ここで制作を行ったモデルがそれ以外の議論とどのように関係し、今後どのような可能性を有するのかについて述べる。
\newpage

%----------------------------------------------------------------------
% 2章 関連研究
%----------------------------------------------------------------------
\chapter{関連研究}
手を認識し、ジェスチャー操作を行うことに関する論文は、HCIの分野に多くある。これらはジェスチャーに関係している。しかし、本研究はジェスチャーに関する研究ではない。
設計論に関する先行研究としてはOOがある。

\section{Felsによる身体性のカテゴリー}
\section{title}
\section{title}
\section{本研究の位置付け}
しかしこれらでは、OOという観点を見落としているのではないか。それを扱う上で、本研究の\textit{grasp}というコンセプトを提示する。
graspとは、OOである。
\newpage

%----------------------------------------------------------------------
% 3章 システム開発/分析手法
%----------------------------------------------------------------------
\chapter{\textit{Grasp}について}
\label{define_model}

\section{概要}
\textit{grasp}は「把握」を意味する動詞・名詞だが、ここでの\textit{grasp}とは「人がある対象に注意や目的意識を抱き、注意を向けた対象を確かめるため、または目的の達成に向けて試行を行う期間」とする。手指の構造を変化させる本作は、このコンセプトに注目し、手指の構造の変換をきっかけに起こる注意や、変換した手指とボールとの関係のもとで生じる目的意識が、体験者の中で自発的に生まれることを目指した。本章では、この\textit{grasp}についての詳細を示す。

\textit{grasp}には二つの性質がある。一つは、時間幅が注意を向けている対象によって、長い場合と短い場合があることである。目的意識が芽生えてからスムーズに操作できるようになる場合、'reach'から'manipulate'が近く、\textit{grasp}は短い、すなわち直感的で使いやすいものとして経験される。その一方、目的意識が芽生えてから試行錯誤を伴い、習熟に長い期間を要する場合、\textit{grasp}は長く、もどかしさを経験し、操れるようになった時に達成感を経験する。

二つ目に、\textit{grasp}の過程で他のことに対する意識が次々と芽生えることがある。具体的には「やってみるまでわからない」といった経験や、物事に対する解像度が高まる中で、当初とは異なる意識が芽生える状況に相当する。

このコンセプトを展開し、「試行錯誤の余地」を設計することを目指したのが、修士作品「Grasp(er)」である。この作品では、\textit{grasp}を経験する中で、個人による創造的な活動が生まれ、\textit{grasp}という動作を行っているのではなく、そこに'er'の接尾辞がついた'Grasper'であると名付けた。

'Grasper'は、「Familiar / Strange」と「Relation」という二つから構成された作品群である。これらに共通する目的は、\textit{grasp}の中で個人が目的意識や興味を抱くことで、\textit{grasp}が連鎖的に生じることである。制作者は「このようなことをしてほしい」という行為の中身を設計せず、体験する個人がその中で次々と注目する対象を見出すことによって、個人が行為を創造していくことを目指している。

このため、「手指の構造や、手指を取り巻く環境を変化させることで、手指の運動に注目する構造」を作ることに取り組んだ。このときの「注目」が起点となって\textit{grasp}の期間が生じるが、その先々で起こる体験は個人に委ねられ、明示的な目的は設定されていない。

この作品群の筐体の設計は、画面の中に出力される図像が「自分の手指に替わる存在」として認識されるように意図されている。体験中に意識が集中できるような構成が目指され、手指を動かす際にはできるだけ自由に動かせるようになっている。

\newpage

%----------------------------------------------------------------------
% 4章 実験結果 
%----------------------------------------------------------------------
\chapter{\textit{Grasp}について}
\label{graspについて}

\section{概要}
第\ref{about_grasper}章では、作品概要と、そこに至るまでのプロトタイピングの分析を通して、作品形態について説明した。

第\ref{graspについて}章では、本研究が提示するコンセプトである\textit{grasp}について詳細に説明する。その上で、Felsの分類との関係性と、この概念を用いることから修士作品の体験についてのねらいを説明する。

\section{\textit{grasp}の定義}
\label{grasp_difinition}
\textit{grasp}について、本研究では次のように定義する。

\begin{quote}
  \textbf{人と対象との関係の中で、人が対象の中に注意や目的意識を抱きながら、意識的に試行する期間}
\end{quote}

graspは「把握」を意味する動詞でもあるが、ここで「動作」ではなく「期間」とした。その理由は、「意識的に試行する」とき、同時にその結果を受けて気づきを得たり、その気づきをもとに新たな関心を抱くといった、単に自分が行為しているだけではなく、対象から影響を受けながら次の行為が決まってくるようなフィードバックループの構造があると考えるためである。「grasp=把握」という言葉についても、単に「ものを掴む」という意味だけでなく、「理解」の意味があることは、対象について一方向に働きかけているのではない様子が現れているのではないだろうか。

\textit{grasp}とは例えば、ギターの習得過程において、弾きこなしたいフレーズを定め、それを達成するまでに試行錯誤をし、達成できるようになるまでの期間である。熟達した状態では、熟達する前とは違う視座でものごとを捉えられるようになり、また違う対象に注意が向くようになると、ギターと人との間に別の\textit{grasp}が芽生える。
またあるいは、\textit{grasp}の過程で、対象と向き合い続ける過程の中でその解像度が高まり、当初目指していたこととは違うことに興味を抱く(セレンディピティ)ことでも、別の\textit{grasp}が芽生える。

ここまで具体例を通してみてきたように、\textit{grasp}は、ギターと人との関係において一度だけ生じるのではなく、注目する対象が定まれば何度でも生じる。
% ここで、\textit{grasp}は人と対象の関係の中で、人が注意を向ける「対象」ごとに別の\textit{grasp}があると言えそうだが、注意を向けている「対象」がなんであるか、断言できなかったり、本人も判然としないこともある。そのため、

\section{Felsの議論との関係性}
ここでは\textit{grasp}が、FelsのEmbodimentとどう関係するのかについて説明する。
先の節\ref{grasp_difinition}に挙げたように、人は\textit{grasp}を通して、あるいはその過程で、対象から影響を受けながら次の試行が形作られていく。この時の「試行」には、挙動を確かめるような動作、すなわちFelsの「Response」もあれば、試行を通して得られた結果をもとに、何かに考えを巡らす「Contemplation」も含まれる。こうした試行の積み重なりから、人と対象のあいだに「Control」や「Belonging」の関係、すなわちFelsの意味でのEmbodimentが生じるのではないか。

つまりこの概念は、折り合いをつけていくまでの「間」に行われていることについて語ることを可能にするものである。\textit{grasp}の様相を捉えることで、現状からEmbodimentが達成されるまでのあいだに欠けているものが何であるかを語る術を得られるのではないか、と考える。

\section{\textit{grasp}を踏まえたIntimacyが起こるまでの過程についての仮説}
\textit{grasp}が、本研究の関心である「対象からの影響も受けつつ(Belonging)、相互の折り合いをつけながら生まれる一体感(Intimacy)が起こる」までの過程に必要ではないか、と考える。

そしてそのためには、注意を向ける対象や目的意識を抱く対象が変わりながらも\textit{grasp}が継続していく体験が良いのか、それとも注意を向ける対象は変わらず、1つのものに対する目的意識が長く継続し、\textit{grasp}も長い体験が良いのか。このいずれであるかが判別することができれば、より詳細にこの体験を説明することができると考えた。

\section{\textit{grasp}を用いた《Grasp(er)》の説明}
本作品を\textit{grasp}の観点から、どのような関係が芽生えることがそのねらいとしてあるかについて、それぞれの作品について説明する。

\subsection*{Familiar / Strange(執筆中)}
後に出てくる「Relation」に比べると、\textit{grasp}の時間は短く、複雑度も低いため、この過程で体験者独自の試行が行われることは多くはない。しかし学内での展示に際して、徐々に変換が複雑になる中で、「ある形を作ってみようとする」ことや「途中から追いつけなくなった法則性を確かめる」ように手指を動かす方がいた。

\subsection*{Relation(執筆中)}
ボールという直接的に制御することができない対象とのあいだでのGraspを経験するので、\textit{grasp}の時間は長い。その過程で「左右に転がしてみる」、「投げ上げる」、「受け止める」といったことを試行する、前の作品よりも注意の向く対象が多い作品である。そして、このとき「自発的に芽生えた目的意識」こそが、制作者の意図を超えたものであると考える。

% \textit{grasp}には二つの性質がある。一つは、時間幅が注意を向けている対象によって、長い場合と短い場合があることである。目的意識が芽生えてからスムーズに操作できるようになる場合、'reach'から'manipulate'が近く、\textit{grasp}は短い、すなわち直感的で使いやすいものとして経験される。その一方、目的意識が芽生えてから試行錯誤を伴い、習熟に長い期間を要する場合、\textit{grasp}は長く、もどかしさを経験し、操れるようになった時に達成感を経験する。\\

% 二つ目に、\textit{grasp}の過程で他のことに対する意識が次々と芽生えることがある。具体的には「やってみるまでわからない」といった経験や、物事に対する解像度が高まる中で、当初とは異なる意識が芽生える状況に相当する。\\

% このコンセプトを展開し、「試行錯誤の余地」を設計することを目指したのが、修士作品《Grasp(er)》である。この作品では、\textit{grasp}を経験する中で、個人による創造的な活動が生まれ、\textit{grasp}という動作を行っているのではなく、そこに'er'の接尾辞がついた'Grasper'であると名付けた。
\newpage

%----------------------------------------------------------------------
% 5章 結論
%----------------------------------------------------------------------
\chapter{バリデーション}
\section{概要}
\section{結果}

\newpage


%----------------------------------------------------------------------
% 謝辞
%----------------------------------------------------------------------
\chapter*{謝辞}
清々しい気持ちでこのページを書いているつもりが、書いてみてようやく自分の至らない点や、自分自身まだわかりきっていない部分が露呈して、実際には軽く吐き気がしています。しかしそれらを整理して再構成するには時間があまりに足りないので、論文執筆を通して指導いただいたことをもとに、時々思い返しては自分自身で赤入れをしていこうと思います。\\
主査の桑久保 亮太教授、副査の平林 真実教授には、度重なるアドバイスと作品・論文の審査をしていただきました。概念を曖昧にして用いていた点についての指摘を受けて論点を整理していく中で、いっそう明確になりました。\\
主指導教員の小林 茂教授には、研究内容にかかわるアドバイスに加え、論文執筆に際してはパラグラフごとのまとまりとパラグラフ間のつながりについて、細かく指導していただきました。応えることのできなかった指摘も多いとは思いますが、教わった作法は今後しっかりと身体化(embodiment)させていきたいと思います。\\
小林ゼミの小南菜子さん、太向弘明さん、成瀬陽太さん、そして聴講の松井美緒さんには、素朴な疑問から鋭い指摘まで、同じ学生同士だからこその忖度のない、貴重な意見をたくさんいただきました。私は今年で卒業ですが、この経験をみなさんの研究活動に還元していくことができたらと思います。また合宿にもいきましょう\footnote{それと、論文執筆に追われていたため未遂に終わった忘年会をやりたいです。}。\\
最後に、インタビューにご協力いただいた皆様、展示の機会を設けてくださったFab Cafe Nagoya様、IAMAS22期生の皆様、そのほか多くの研究にご協力いただいた皆様\footnote{列挙したらキリがなく、ひとまとめにしてすみません。これを執筆している今、お世話になった人々が、走馬灯のように頭の中を駆け回っています。あなたのことを、忘れていません。お会いしたときに直接感謝を伝えさせてください。}に深く感謝申し上げます。

\addcontentsline{toc}{chapter}{謝辞}
\newpage

%----------------------------------------------------------------------
% 参考文献
%----------------------------------------------------------------------
% \input{contents/references.tex}
\bibliographystyle{sty/sieicej}
\bibliography{references}

%----------------------------------------------------------------------
% 付録
%----------------------------------------------------------------------
\newpage
\appendix
\include{Appendix_model}

% \label{付録}
% \input{contents/appendix_a.tex}

\end{document}